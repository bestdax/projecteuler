\section{复合多边形数}\label{sec:problem45}
\subsection{问题描述}
\begin{tcolorbox}
	三角形数、五边形数和六边形数由以下公式给出:

	\[
		T_n = \frac{n(n + 1)}{2}
	\]
	\[
		P_n = \frac{n(3n - 1)}{2}
	\]
	\[
		H_n = n(2n - 1)
	\]

	可以发现, $ T_{285} = P_{165} = H_{143} =40755 $。

	找出下一个既是三角形数、五边形数、也是六边形数的数。

\end{tcolorbox}

\subsection{算法}
可以从 $ H_{144}$开始查找,用三角形数和五边形数的判断函数。

\begin{align*}
	T_n        & = \frac{n(n+1)}{2}                      \\
	2T_n       & = n^2 + n                               \\
	8T_n       & = 4n^2 + 4n + 1 - 1                     \\
	(2n + 1)^2 & = 8T_n + 1                              \\
	n          & = \frac{\sqrt{8T_n + 1} - 1}{2} \tag{1}
\end{align*}

\begin{align*}
	P_n        & = \frac{n(3n - 1)}{2}                    \\
	2P_n       & = 3n^2 - n                               \\
	24P_n      & = 36n^2 - 12n + 1 - 1                    \\
	(6n - 1)^2 & = 24P_n + 1                              \\
	n          & = \frac{\sqrt{24P_n + 1} + 1}{6} \tag{2}
\end{align*}
六边形的判断其实这道题里面用不到,不过这里还是列一下算法:
\begin{align*}
	H_n        & = n(2n - 1)                             \\
	H_n        & = 2n^2 - n                              \\
	8H_n       & = 16n^2 - 8n + 1 - 1                    \\
	(4n - 1)^2 & = 8H_n + 1                              \\
	n          & = \frac{\sqrt{8H_n + 1} + 1}{4} \tag{3}
\end{align*}
\subsection{答案}
1533776805
