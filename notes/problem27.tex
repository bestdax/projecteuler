\section{二次表达式生成的素数}
\subsection{问题描述}
\begin{tcolorbox}
欧拉发现了这个著名的二次表达式:

\[
n^2 + n + 41
\]

对于 \( n = 0 \) 到 \( n = 39 \) 之间的每个 \( n \),该表达式生成了 40 个素数。然而,当 \( n = 40 \) 时,得到的 \( 40^2 + 40 + 41 = 1681 \) 并不是一个素数。该二次表达式生成了从 \( n = 0 \) 到 \( n = 39 \) 的连续素数。

发现了以下的二次表达式:

\[
n^2 - 79n + 1601
\]

对于 \( n = 0 \) 到 \( n = 79 \) 之间的每个 \( n \),该表达式生成了 80 个素数。这一表达式生成的连续素数的系数积为 \( -79 \) 和 1601,积为 \( -79 \times 1601 = -126479 \)。

考虑形如 \( n^2 + an + b \) 的二次表达式,其中 \( |a| < 1000 \) 且 \( |b| \leq 1000 \)。

找出系数 \( a \) 和 \( b \) 使得表达式 \( n^2 + an + b \) 对于从 \( n = 0 \) 开始的连续整数 \( n \) 生成最多的素数,计算出此时 \( a \) 和 \( b \) 的乘积。

\end{tcolorbox}

\subsection{算法}
最简单粗暴的方法是直接两遍循环,但是这种方法的消耗是指数级增长的。我们可以对暴力循环进行一些优化。

\begin{itemize}
  \item 首先,我们来考虑 \( b \),当 \( n = 0 \),二项式的值为 \( b \),所以 \( b \)必须为素数才行;
  \item \( a \)的值比较自由,但是也有可以考虑的点,比如当 \( n = 1 \)时,二项式的值为 \( 1 + a + b
    \),除了值等于2的情况,其他情况下, \( a \)和 \( b \)的奇偶性必须是相同的。但是因为 \( b \)必须为素数,所以也排除了
    \( n = 1 \)时二项式的值为2的情况,所以 \( a \)和 \( b \)同奇或者同偶。
\end{itemize}

\subsection{答案}
