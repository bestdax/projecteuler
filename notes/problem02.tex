\section{Even Fibonacci numbers}
\subsection{Description}
Each new term in the Fibonacci sequence is generated by adding the previous two terms. By starting with 1 and 2, the first 10 terms will be:
\begin{equation*}
	1, 2, 3, 5, 8, 13, 21, 34, 55, 89, \dots
\end{equation*}

By considering the terms in the Fibonacci sequence whose values do not exceed four million, find the sum of the even-valued terms.

\subsection{Flow Chart}
\begin{center}
	\begin{tikzpicture}[node distance=2cm]
		\node (start) [startstop] {开始};
		\node (input) [io, right of=start, xshift=4cm] {输入一个上限值cap};
		\node (op1) [operation, below of=input] {a = 0; \\
			b = 1\\
			sum = 0;};
		\node (op2) [operation, below of=op1] {c = a + b;\\
			a = b;\\
			b = c;};
		\node (dec1) [decision, below of=op2, yshift=-.5cm] {c \% 2 == 0};
		\node (op3) [operation, below of=dec1, yshift=-.3cm] {sum += c;};
		\node (dec2) [decision, below of=op3] {c < cap};
		\node (output) [io, below of=dec2] {输出答案};
		\node (end) [startstop, left of=output, xshift=-4cm] {结束};

		\draw [arrow] (start) -- (input);
		\draw [arrow] (input) -- (op1);
		\draw [arrow] (op1) -- (op2);
		\draw [arrow] (op2) -- (dec1);
		\draw [arrow] (dec1) -- node[right]{Yes} (op3);
		\draw [arrow] (op3) --  (dec2);
		\draw [arrow] (dec2) -- node[right]{No} (output);
		\draw [arrow] (dec2.west) -| node[above=3cm, xshift=-.5cm] {Yes}([xshift=-2cm]op2.west) --  (op2.west);
		\draw [arrow] (output) --  (end);
	\end{tikzpicture}
	\captionof{figure}{Sum Of Even Fibonacci}
\end{center}

\subsection{Codes}
\begin{cpp}
	std::vector<int> Solution::fibs(int cap)
	{
			std::vector<int> fb;
			int a = 0, b = 1, c = 1;

			while(c <= cap)
			{
					fb.push_back(c);
					a = b;
					b = c;
					c = a + b;
				}

			return fb;
		}

	long Solution::sum_of_even_fibs(int cap)
	{
			long sum{};
			std::vector<int> fb = fibs(cap);

			for(auto item : fb)
			{
					if(item % 2 == 0) sum += item;
				}

			return sum;
		}
\end{cpp}
