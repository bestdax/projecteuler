\section{数列中的最大乘积}\label{sec:problem08}
\subsection{问题描述}
\begin{tcolorbox}
在下面这个 1000 位的数字中,相邻的四个数字的最大乘积是 9 × 9 × 8 × 9 = 5832。

\begin{verbatim}
73167176531330624919225119674426574742355349194934
96983520312774506326239578318016984801869478851843
85861560789112949495459501737958331952853208805511
12540698747158523863050715693290963295227443043557
66896648950445244523161731856403098711121722383113
62229893423380308135336276614282806444486645238749
30358907296290491560440772390713810515859307960866
70172427121883998797908792274921901699720888093776
65727333001053367881220235421809751254540594752243
52584907711670556013604839586446706324415722155397
53697817977846174064955149290862569321978468622482
83972241375657056057490261407972968652414535100474
82166370484403199890008895243450658541227588666881
16427171479924442928230863465674813919123162824586
17866458359124566529476545682848912883142607690042
24219022671055626321111109370544217506941658960408
07198403850962455444362981230987879927244284909188
84580156166097919133875499200524063689912560717606
05886116467109405077541002256983155200055935729725
71636269561882670428252483600823257530420752963450
\end{verbatim}

找出这个 1000 位数字中,相邻的 13 个数字的乘积最大的值。这个最大的乘积是多少?
\end{tcolorbox}

\subsection{算法}
这道题比较简单,只要遍历即可。如果遇到0就直接跳到0后面的数字再开始。

需要注意的是这个算法中会用字符串来表示大数字,所以取出的数字是字符,需要做相应的运算才能成为数值。

\subsection{答案}
23514624000
