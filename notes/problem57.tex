\section{方根的渐进分数}
\subsection{问题描述}
\begin{tcolorbox}
	可以证明,$\sqrt{2}$ 可以表示为一个无限的连分数:
	\[
		\sqrt{2} = 1 + \cfrac{1}{2 + \cfrac{1}{2 + \cfrac{1}{2 + \dots}}}
	\]

	通过展开前四次迭代,我们得到:
	\[
		1 + \cfrac{1}{2} = \cfrac{3}{2} = 1.5
	\]
	\[
		1 + \cfrac{1}{2 + \cfrac{1}{2}} = \cfrac{7}{5} = 1.4
	\]
	\[
		1 + \cfrac{1}{2 + \cfrac{1}{2 + \cfrac{1}{2}}} = \cfrac{17}{12} = 1.41666 \dots
	\]
	\[
		1 + \cfrac{1}{2 + \cfrac{1}{2 + \cfrac{1}{2 + \cfrac{1}{2}}}} = \cfrac{41}{29} = 1.41379 \dots
	\]

	在第八次展开时,$\cfrac{1393}{985}$ 是第一个分子位数超过分母位数的例子。

	问题:在前 1000 次展开中,有多少个分数的分子的位数多于分母的位数?
\end{tcolorbox}

\subsection{算法}
这道题涉及到连分数的知识, $ \sqrt{2} $的连分数可以表示成 $ \langle 1, \overline{2} \rangle $, 分子分母的递推公式如下:
\begin{align*}
	P_n & = a_nP_{n-1} + P_{n-2} \\
	Q_n & = a_nQ_{n-1} + Q_{n-2} \\
\end{align*}
其中 \( P_{-1} = 1, P_{-2} = 0, Q_{-1} = 0, Q_{-2} = 1 \)

当然这道题还是要用到大整数,1000次展开将是一个很大的数。

\subsection{答案}
153
