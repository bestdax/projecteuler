\chapter{Square Root Convergents}
\section{Description}

It is possible to show that the square root of two can be expressed as an infinite continued fraction.
\[
	\sqrt{2} = 1 + \frac{1}{2 + \frac{1}{2 + \frac{1}{2 + \dots}}}
\]




By expanding this for the first four iterations, we get:
\begin{align*}
	1 + \frac{1}{2} = \frac{3}{2}                                             & = 1.5          \\
	1 + \frac{1}{2 + \frac{1}{2}} = \frac{7}{5}                               & = 1.4          \\
	1 + \frac{1}{2 + \frac{1}{2 + \frac{1}{2}}} = \frac{17}{12}               & = 1.41666\dots \\
	1 + \frac{1}{2 + \frac{1}{2 + \frac{1}{2 + \frac{1}{2}}}} = \frac{41}{29} & = 1.41379\dots \\
\end{align*}

The next three expansions are $\frac{99}{70}$, $\frac{239}{169}$, and $\frac{577}{408}$, but the eighth expansion,
$\frac{1393}{985}$, is the first example where the number of digits in the numerator exceeds the number of digits in the
denominator.

In the first one-thousand expansions, how many fractions contain a numerator with more digits than the denominator?
