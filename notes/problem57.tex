\section{Square Root Convergents}
\subsection{Description}

It is possible to show that the square root of two can be expressed as an infinite continued fraction.
\begin{equation*}
	\sqrt{2} = 1 + \cfrac{1}{2 + \cfrac{1}{2 + \cfrac{1}{2 + \dots}}}
\end{equation*}

By expanding this for the first four iterations, we get:
\begin{align*}
	1 + \dfrac{1}{2} = \dfrac{3}{2}                                                & = 1.5          \\
	1 + \cfrac{1}{2 + \cfrac{1}{2}} = \cfrac{7}{5}                                 & = 1.4          \\
	1 + \cfrac{1}{2 + \cfrac{1}{2 + \cfrac{1}{2}}} = \cfrac{17}{12}                & = 1.41666\dots \\
	1 + \cfrac{1}{2 + \cfrac{1}{2 + \cfrac{1}{2 + \cfrac{1}{2}}}} = \cfrac{41}{29} & = 1.41379\dots \\
\end{align*}

The next three expansions are \( \dfrac{99}{70} \), \( \dfrac{239}{169} \), and \( \dfrac{577}{408} \), but the eighth expansion,
\( \dfrac{1393}{985} \), is the first example where the number of digits in the numerator exceeds the number of digits in the
denominator.

In the first one-thousand expansions, how many fractions contain a numerator with more digits than the denominator?
