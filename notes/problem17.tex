\section{数字文字计数}\label{sec:problem17}
\subsection{问题描述}
\begin{tcolorbox}
	如果将 1 到 5 写成英文单词,它们分别是:

	\begin{itemize}
		\forcsvlist{\item}{
			1 = \enquote{one},
			2 = \enquote{two},
			3 = \enquote{three},
			4 = \enquote{four},
			5 = \enquote{five},
		      }
	\end{itemize}
	这些单词的字母数量为 3、3、5、4 和 4。因此,1 到 5 的数字单词一共有
	$ 3 + 3 + 5 + 4 + 4 = 19 $
	个字母。

	从 1 到 1000(含 1000)中的每个数字用英文单词来表示,不计空格和连字符,总共有多少个字母?

	例如:
	342(three hundred and forty-two)包含 23 个字母。

	115(one hundred and fifteen)包含 20 个字母。

	注意:不计空格或连字符,\enquote{and} 是按规则包含在使用中的英文单词中。
\end{tcolorbox}

\subsection{算法}
\begin{enumerate}
	\item 准备数字到英文单词的映射表:我们需要将 1 到 19 的单词直接映射。对于整十数(20, 30, ...,
		90),也要单独映射。然后为\enquote{hundred}(100)和\enquote{thousand}(1000)定义规则。
	\item 分段处理数字:例如,对于 342,英文单词表示为 \textit{three hundred and forty-two}。我们可以将数字拆分为百位数、十位数、个位数,分别进行转换。
	\item 处理特殊情况:例如 100 到 999 之间的数字,需要注意添加 \enquote{and}。
	\item 统计字母数:统计所有数字转换成单词后的字母数量,不计空格和连字符。
\end{enumerate}

\subsection{答案}
21124
