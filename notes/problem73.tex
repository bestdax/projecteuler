\section{范围内的分数个数}\label{sec:problem73}
\subsection{问题描述}
\begin{tcolorbox}
	考虑分数 \( \frac{n}{d} \),其中 \( n \) 和 \( d \) 是正整数。如果 \( n < d \) 并且 \( \text{HCF}(n, d) = 1 \),它被称为简化真分数。

	如果我们列出 \( d \leq 8 \) 的所有简化真分数,并按大小升序排列,我们得到:
	\[ \frac{1}{8}, \frac{1}{7}, \frac{1}{6}, \frac{1}{5}, \frac{1}{4}, \frac{2}{7}, \frac{1}{3}, \frac{\bf 3}{\bf 8},
		\frac{\bf 2}{\bf 5}, \frac{\bf 3}{\bf 7}, \frac{1}{2}, \frac{4}{7}, \frac{3}{5}, \frac{5}{8}, \frac{2}{3}, \frac{5}{7}, \frac{3}{4}, \frac{4}{5}, \frac{5}{6}, \frac{6}{7}, \frac{7}{8} \]

	可以看出,在 \( \frac{1}{3} \) 和 \( \frac{1}{2} \) 之间有 3 个分数。

	在 \( d \leq 12000 \) 的所有简化真分数的排序集合中,有多少个分数位于 \( \frac{1}{3} \) 和 \( \frac{1}{2} \) 之间?
\end{tcolorbox}

\subsection{算法}
跟问题71中的解法类似,只是多了一个边界而已。

\subsection{答案}
7295372
