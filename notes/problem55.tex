\section{Lychrel数字}\label{sec:problem55}
\subsection{问题描述}
\begin{tcolorbox}
	如果我们取 $47$,反转并相加,得到 $47 + 74 = 121$,这是一个回文。

	并非所有数字都能如此快速产生回文。例如,

	\begin{align}
		349 + 943   & = 1292 \\
		1292 + 2921 & = 4213 \\
		4213 + 3124 & = 7337
	\end{align}

	也就是说,$349$ 经过三次迭代才得到回文。

	虽然尚未证明,但人们认为某些数字(如 $196$)永远不会产生回文。一个通过反转和相加过程永远无法形成回文的数字称为“Lychrel 数字”。由于这些数字的理论性质,为了本问题的目的,我们假设一个数字是 Lychrel 直到被证明不是。此外,您被告知,对于每个小于一万的数字,它将要么
	\begin{enumerate}[label={(\roman*)}]
		\item 在少于五十次迭代中成为回文,或
		\item 没有人能用现有的计算能力将其映射到回文。
	\end{enumerate}
	事实上,$10677$ 是第一个被证明需要超过五十次迭代才能产生回文的数字:$4668731596684224866951378664$ ($53$ 次迭代,$28$ 位数)。

	令人惊讶的是,有些回文数字本身也是 Lychrel 数字;第一个例子是 $4994$。

	在一万以下有多少个 Lychrel 数字?
\end{tcolorbox}

\subsection{算法}
因为有可能会溢出,所以用bigint类来处理。

\subsection{答案}
249
