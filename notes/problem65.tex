\section[\lower{e}的连分数]{$e$的连分数}
\subsection{问题描述}
\begin{tcolorbox}[breakable]

	2 的平方根可以表示为一个无限连分数。

	\[ \sqrt{2} = 1 + \dfrac{1}{2 + \dfrac{1}{2 + \dfrac{1}{2 + \cdots}}} \]

	这个无限连分数可以写作:

	\[ \sqrt{2} = [1; (2)] \]

	其中 \((2)\) 表示 2 无限循环。类似地,\(\sqrt{23} = [4; (1, 3, 1, 8)]\)。

	结果表明,连分数的部分值序列提供了平方根的最佳有理数近似。让我们考虑 \(\sqrt{2}\) 的连分数近似值。

	\[ 1 + \dfrac{1}{2} = \dfrac{3}{2} \]

	\[ 1 + \dfrac{1}{2 + \dfrac{1}{2}} = \dfrac{7}{5} \]

	\[ 1 + \dfrac{1}{2 + \dfrac{1}{2 + \dfrac{1}{2}}} = \dfrac{17}{12} \]

	\[ 1 + \dfrac{1}{2 + \dfrac{1}{2 + \dfrac{1}{2 + \dfrac{1}{2}}}} = \dfrac{41}{29} \]

	因此,\(\sqrt{2}\) 的前十个连分数近似值序列是:

	1,
	\[ \dfrac{3}{2}, \dfrac{7}{5}, \dfrac{17}{12}, \dfrac{41}{29}, \dfrac{99}{70}, \dfrac{239}{169}, \dfrac{577}{408}, \dfrac{1393}{985}, \dfrac{3363}{2378}, \ldots \]

	最令人惊讶的是,重要的数学常数 \( e \) 可以表示为:

	\[ e = [2; 1, 2, 1, 1, 4, 1, 1, 6, 1, \ldots, 1, 2k, 1, \ldots] \]

	\( e \) 的前十个连分数近似值序列是:

	2,
	\[ \dfrac{3}{1}, \dfrac{8}{3}, \dfrac{11}{4}, \dfrac{19}{7}, \dfrac{87}{32}, \dfrac{106}{39}, \dfrac{193}{71}, \dfrac{1264}{465}, \dfrac{1457}{536}, \ldots \]

	第 10 个连分数的分子的数字之和是 \(1 + 4 + 5 + 7 = 17\)。

	找到 \( e \) 的连分数的第 100 个连分数的分子的数字之和。
\end{tcolorbox}

\subsection{算法}

\subsection{答案}
