\section{Problem 65 Convergents of e}
\subsection{Description}
The square root of 2 can be written as an infinite continued fraction.
\begin{equation*}
	\sqrt{2} = 1 + \cfrac{1}{2 + \cfrac{1}{2 + \cfrac{1}{2 + \cfrac{1}{2 + ...}}}}
\end{equation*}

The infinite continued fraction can be written,\( \sqrt{2} = [1; (2)] \), indicates that 2 repeats ad infinitum. In a
similar way, \( \sqrt{23} = [4; (1, 3, 1, 8)] \).

It turns out that the sequence of partial values of continued fractions for square roots provide the best rational
approximations. Let us consider the convergents for \( \sqrt{2} \).

\begin{align*}
	 & 1 + \dfrac{1}{2} = \dfrac{3}{2}                                                \\
	 & 1 + \cfrac{1}{2 + \cfrac{1}{2}} = \dfrac{7}{5}                                 \\
	 & 1 + \cfrac{1}{2 + \cfrac{1}{2 + \cfrac{1}{2}}} = \dfrac{17}{12}                \\
	 & 1 + \cfrac{1}{2 + \cfrac{1}{2 + \cfrac{1}{2 + \cfrac{1}{2}}}} = \dfrac{41}{29}
\end{align*}

Hence the sequence of the first ten convergents for \( \sqrt{2} \) are:

\begin{equation*}
	1, \dfrac{3}{2}, \dfrac{7}{5}, \dfrac{17}{12}, \dfrac{41}{29}, \dfrac{99}{70}, \dfrac{239}{169}, \dfrac{577}{408},
	\dfrac{1393}{985}, \dfrac{3363}{2378}, \dots
\end{equation*}

What is most surprising is that the important mathematical constant,
\begin{equation*}
	e = [2; 1, 2, 1, 1, 4, 1, 1, 6, 1, ... , 1, 2k, 1, ...]
\end{equation*}

The first ten terms in the sequence of convergents for e are:
\begin{equation*}
	2, 3, \dfrac{8}{3}, \dfrac{11}{4}, \dfrac{19}{7}, \dfrac{87}{32}, \dfrac{106}{39}, \dfrac{193}{71}, \dfrac{1264}{465},
	\dfrac{1457}{536}, \dots
\end{equation*}

The sum of digits in the numerator of the \( 10^{th} \) convergent is \( 1 + 4 + 5 + 7 = 17 \) .

Find the sum of digits in the numerator of the \( 100^{th} \) convergent of the continued fraction for.
