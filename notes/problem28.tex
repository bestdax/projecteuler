\section{螺旋矩阵对角线之和}
\subsection{问题描述}
\begin{tcolorbox}

	从数字1开始,按顺时针顺序依次将连续的奇数排列成一个边长为 5 的螺旋矩阵如下所示:

	\[
		\begin{matrix}
			\red{21} & 22      & 23      & 24      & \red{25} \\
			20       & \red{7} & 8       & \red{9} & 10       \\
			19       & 6       & \red{1} & 2       & 11       \\
			18       & \red{5} & 4       & \red{3} & 12       \\
			\red{17} & 16      & 15      & 14      & \red{13}
		\end{matrix}
	\]

	可以看到,该矩阵对角线上的数字之和为 101。

	考虑更大的边长为 1001 的螺旋矩阵,求该矩阵对角线上的数字之和。
\end{tcolorbox}

\subsection{算法}
对角线数字的生成方法为:
\begin{itemize}
	\item 从1开始,第二层矩阵的对角线四个角点分别加2;
	\item 下一层的矩阵的步长要加2;
\end{itemize}

\subsection{答案}
