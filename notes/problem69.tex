\section{最大欧拉商数}\label{sec:problem69}
\subsection{问题描述}
\begin{tcolorbox}
	\textbf{欧拉的欧拉函数} \(\phi(n)\)(有时也称为phi函数),定义为不超过 \(n\) 的正整数中与 \(n\) 互质的数的数量。例如,1、2、4、5、7 和 8 都小于或等于9且与9互质,所以 \(\phi(9)=6\)。

	\begin{center}
		\begin{tabular}{|c|c|c|c|}
			\hline
			\(n\) & 互质数      & \(\phi(n)\) & \(n/\phi(n)\) \\
			\hline
			2     & 1           & 1           & 2             \\
			3     & 1,2         & 2           & 1.5           \\
			4     & 1,3         & 2           & 2             \\
			5     & 1,2,3,4     & 4           & 1.25          \\
			6     & 1,5         & 2           & 3             \\
			7     & 1,2,3,4,5,6 & 6           & 1.1666...     \\
			8     & 1,3,5,7     & 4           & 2             \\
			9     & 1,2,4,5,7,8 & 6           & 1.5           \\
			10    & 1,3,7,9     & 4           & 2.5           \\
			\hline
		\end{tabular}
	\end{center}

	可以看出,当 \(n=6\) 时,\(n/\phi(n)\) 在 \(n \leq 10\) 中达到最大值。

	找出 \(n \leq \num{1000000}\) 时,\(n/\phi(n)\) 达到最大值的 \(n\)。
\end{tcolorbox}

\subsection{算法}
欧拉函数求法
\begin{align*}
	\varphi (n) & =\varphi \left(\prod _{i=1}^{r}p_{i}^{k_{i}}\right)  \\
	            & =\prod _{i=1}^{r}\varphi \left(p_{i}^{k_{i}}\right)
	            & {\text{( 积 性 )}}                                 \\
	            & =\prod _{i=1}^{r}p_{i}^{k_{i}-1}(p_{i}-1)
	            & {\text{( 质 数 幂 处 取 值 )}}                     \\
	            & =n\prod _{i=1}^{r}\left(1-{\frac {1}{p_{i}}}\right).
\end{align*}

但是这道题涉及到大量求解 \varphi 值的情况,所以需要用筛法来解决。

在用欧拉线性筛法的过程,一个合数总是被其最小的质因子筛除,比如合数 \( c = n  \times p\), \( p \)是   \( c
\)的最小质因数。这时候可以分两种情况来讨论:
\begin{enumerate}
	\item \( n \)与 \( p \)互质,则 \( \varphi(c) = \varphi(p)\varphi(n) \)
	\item \( n \equiv 0 \pmod{p} \), 则 \( n \)包含了 \( c \)的所有质因子。根据欧拉函数的公式:
	      \begin{align*}
		      \varphi(c) & = c \sum_{i=1}^{r}(1 - \frac{1}{p_i}) \\
		                 & = np \sum_{i=1}^{r}(1- \frac{1}{p_i}) \\
		                 & = p \varphi(n)
	      \end{align*}
\end{enumerate}

\subsection{答案}
510510
