\section{Totient Maximum}
\subsection{Problem Description}
\begin{tcolorbox}
	Euler's totient function,  \( \phi(n) \)
	[sometimes called the phi function], is defined as the number  of positive integers not exceeding \( n \)
	which are relatively prime to \( n \)
	. For example, as \( 1,2,4,5,7 \) and \( 8 \) , are all less than or equal to nine and relatively prime to nine,  \(
	\phi(9) = 6 \).

	\begin{center}
		\begin{tabular}{clcl}\toprule
			\( n \) & Relatively  Prime & \( \phi(n) \) & \( n / \phi(n) \) \\ \midrule
			2       & 1                 & 1             & 2                 \\
			3       & 1,2               & 2             & 1.5               \\
			4       & 1,3               & 2             & 2                 \\
			5       & 1,2,3,4           & 4             & 1.25              \\
			6       & 1,5               & 2             & 3                 \\
			7       & 1,2,3,4,5,6       & 6             & 1.1666...         \\
			8       & 1,3,5,7           & 4             & 2                 \\
			9       & 1,2,4,5,7,8       & 6             & 1.5               \\
			10      & 1,3,7,9           & 4             & 2.5               \\ \bottomrule
		\end{tabular}
	\end{center}

	It can be seen that \( n=6 \) produces a maximum \( \phi(n) / n \) for \( n \leqslant 10 \).

	Find the value of \( n \leqslant 1 000 000 \) for which \( \phi(n) / n \) is a maximum.
\end{tcolorbox}
