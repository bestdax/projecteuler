\section{XOR 解密}\label{sec:problem59}
\subsection{问题描述}
\begin{tcolorbox}

	每个计算机上的字符都被分配了一个唯一的代码,优选标准是 ASCII(美国标准信息交换代码)。例如,大写字母 A = 65,星号 (*) = 42,小写字母 k = 107。

	一种现代加密方法是将文本文件转换为 ASCII,然后将每个字节与来自密钥的给定值进行 XOR 操作。使用相同的加密密钥对密文进行操作,可以恢复明文。

	你的任务是解密一段由三个小写字母组成的加密密钥生成的消息,并找出原始文本中 ASCII 值的总和。

	请\href{https://projecteuler.net/resources/documents/0059_cipher.txt}{点击}下载文本。
\end{tcolorbox}

\subsection{算法}
用异或生成文本,然后在文本中搜索\texttt{" the "},要包括两边的空格。

\subsection{答案}
129448
