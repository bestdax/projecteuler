\section{全数字的部分整除性}\label{sec:problem43}
\subsection{问题描述}
\begin{tcolorbox}
	将0到9这十个数字以某种排列组合的形式组成一个10位数,称为全数字数。例如,1406357289 是一个全数字数。

	假设 \( d_1 \) 表示第一个数字,\( d_2 \) 表示第二个数字,依此类推。我们定义如下特性:

	\begin{itemize}
		\item \( d_2d_3d_4 \) 能被 2 整除
		\item \( d_3d_4d_5 \) 能被 3 整除
		\item \( d_4d_5d_6 \) 能被 5 整除
		\item \( d_5d_6d_7 \) 能被 7 整除
		\item \( d_6d_7d_8 \) 能被 11 整除
		\item \( d_7d_8d_9 \) 能被 13 整除
		\item \( d_8d_9d_{10} \) 能被 17 整除
	\end{itemize}

	找到所有满足上述特性的全数字数,并求它们的和。

\end{tcolorbox}

\subsection{算法}
用 \( 0123456789 \)的排列变化来逐项进行判断,但是因为0不能放在首位,所以可以从 \( 1023456789 \)开始。

这种算法要计算出所有的排列,其实效率并不高,但是容易理解。

可以通过剪枝算法逐渐构造出整个10位的全数字数,因为不符合整除的条件的分枝被剪除了,所以避免了很多不必要的运算。

\subsection{答案}
16695334890
