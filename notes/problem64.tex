\section{标题}
\subsection{问题描述}
\begin{tcolorbox}[breakable]
	所有平方根作为连分数表示时都是周期性的,可以写成以下形式:

	\[ \sqrt{N} = a_0 + \cfrac{1}{a_1 + \cfrac{1}{a_2 + \cfrac{1}{a_3 + \cdots}}} \]

	例如,让我们考虑 $\sqrt{23}$:

	\[ \sqrt{23} = 4 + \sqrt{23} - 4 = 4 + \cfrac{1}{\cfrac{1}{\sqrt{23} - 4}} = 4 + \cfrac{1}{1 + \cfrac{\sqrt{23} - 3}{7}} \]

	如果我们继续这个过程,我们会得到以下展开:

	\[ \sqrt{23} = 4 + \cfrac{1}{1 + \cfrac{1}{3 + \cfrac{1}{1 + \cfrac{1}{8 + \cdots}}}} \]

	这个过程可以总结如下:
	\begin{align*}
		a_0 & = 4, & \cfrac{1}{\sqrt{23} - 4}  = & \cfrac{\sqrt{23} + 4}{7}      = 1 + \cfrac{\sqrt{23} - 3}{7} \\
		a_1 & = 1, & \cfrac{7}{\sqrt{23} - 3}  = & \cfrac{7(\sqrt{23} + 3)}{14}  = 3 + \cfrac{\sqrt{23} - 3}{2} \\
		a_2 & = 3, & \cfrac{2}{\sqrt{23} - 3}  = & \cfrac{2(\sqrt{23} + 3)}{14}  = 1 + \cfrac{\sqrt{23} - 4}{7} \\
		a_3 & = 1, & \cfrac{7}{\sqrt{23} - 4}  = & \cfrac{7(\sqrt{23} + 4)}{7}   = 8 + \sqrt{23} - 4            \\
		a_4 & = 8, & \cfrac{1}{\sqrt{23} - 4}  = & \cfrac{\sqrt{23} + 4}{7} = 1 + \cfrac{\sqrt{23} - 3}{7}      \\
		a_5 & = 1, & \cfrac{7}{\sqrt{23} - 3}  = & \cfrac{7(\sqrt{23} + 3)}{14} = 3 + \cfrac{\sqrt{23} - 3}{2}  \\
		a_6 & = 3, & \cfrac{2}{\sqrt{23} - 3}  = & \cfrac{2(\sqrt{23} + 3)}{14} = 1 + \cfrac{\sqrt{23} - 4}{7}  \\
		a_7 & = 1, & \cfrac{7}{\sqrt{23} - 4}  = & \cfrac{7(\sqrt{23} + 4)}{7} = 8 + \sqrt{23} - 4              \\
	\end{align*}

	可以看出序列是重复的。为了简洁,我们使用记号 $\sqrt{23} = [4; (1, 3, 1, 8)]$,表示块 (1, 3, 1, 8) 无限重复。

	前十个连分数表示的(无理数)平方根是:
	\begin{align*}
		\sqrt{2}  & = [1; (2)], \quad \text{周期} = 1             \\
		\sqrt{3}  & = [1; (1, 2)], \quad \text{周期} = 2          \\
		\sqrt{5}  & = [2; (4)], \quad \text{周期} = 1             \\
		\sqrt{6}  & = [2; (2, 4)], \quad \text{周期} = 2          \\
		\sqrt{7}  & = [2; (1, 1, 1, 4)], \quad \text{周期} = 4    \\
		\sqrt{8}  & = [2; (1, 4)], \quad \text{周期} = 2          \\
		\sqrt{10} & = [3; (6)], \quad \text{周期} = 1             \\
		\sqrt{11} & = [3; (3, 6)], \quad \text{周期} = 2          \\
		\sqrt{12} & = [3; (2, 6)], \quad \text{周期} = 2          \\
		\sqrt{13} & = [3; (1, 1, 1, 1, 6)], \quad \text{周期} = 5 \\
	\end{align*}

	对于 $\sqrt{N} \leq 13$,正好有四个连分数表示具有奇数周期。

	对于 $\sqrt{N} \leq 10000$,有多少个连数表示具有奇数周期?

\end{tcolorbox}

\subsection{算法}

\subsection{答案}
