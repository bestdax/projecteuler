\chapter{Odd Period Square Roots}
\section{Problem 64}

All square roots are periodic when written as continued fractions and can be written in the form:

\[
	\sqrt{N} = a_0 + \frac{1}{a_1 + \frac{1}{a_2 + \frac{1}{a3 + \dots}}}
\]


For example, let us consider $\sqrt{23}$:

\begin{align*}
	\sqrt{23} = 4 + \sqrt{23} - 4 = 4 + \frac{1}{\frac{1}{\sqrt{23}-4}} = 4 + \frac{1}{1 + \frac{\sqrt{23} - 3}{7}}
\end{align*}

If we continue we would get the following expansion:

\begin{align*}
	\sqrt{23} = 4 + \frac{1}{1 + \frac{1}{3 + \frac{1}{1 + \frac{1}{8 + \dots}}}}
\end{align*}

The process can be summarised as follows:

\begin{align*}
	a_0 & = 4, \frac{1}{\sqrt{23} - 4}  = \frac{\sqrt{23} + 4}{7}      = 1 + \frac{\sqrt{23} -3}{7} \\
	a_1 & = 1, \frac{7}{\sqrt{23} - 3}  = \frac{7(\sqrt{23} + 3)}{14}  = 3 + \frac{\sqrt{23} -3}{2} \\
	a_2 & = 3, \frac{2}{\sqrt{23} - 3}  = \frac{2(\sqrt{23} + 3)}{14}  = 1 + \frac{\sqrt{23} -4}{7} \\
	a_3 & = 1, \frac{7}{\sqrt{23} - 4}  = \frac{7(\sqrt{23} + 4)}{7}   = 8 + {\sqrt{23} -4}         \\
	a_4 & = 8, \frac{1}{\sqrt{23} - 4}  = \frac{\sqrt{23} + 4}{7}      = 1 + \frac{\sqrt{23} -3}{7} \\
	a_5 & = 1, \frac{7}{\sqrt{23} - 3}  = \frac{7(\sqrt{23} + 3)}{14}  = 3 + \frac{\sqrt{23} -3}{2} \\
	a_6 & = 3, \frac{2}{\sqrt{23} - 3}  = \frac{2(\sqrt{23} + 3)}{14}  = 1 + \frac{\sqrt{23} -4}{7} \\
	a_7 & = 1, \frac{7}{\sqrt{23} - 4}  = \frac{7(\sqrt{23} + 4)}{7}   = 8 + {\sqrt{23} -4}         \\
\end{align*}

It can be seen that the sequence is repeating. For conciseness, we use the notation $\sqrt{23} = [4;(1,3,1,8)]$, to
indicate that the block (1,3,1,8) repeats indefinitely.

The first ten continued fraction representations of (irrational) square roots are:

$\sqrt{2} = [1;(2)]$, period=$1$

$\sqrt{3} = [1;(1, 2)]$, period=$2$

$\sqrt{5} = [2;(4)]$, period=$1$

$\sqrt{6} = [2;(2,4)]$, period=$2$

$\sqrt{7} = [1;(1,1,1,4)]$, period=$4$

$\sqrt{8} = [1;(1,4)]$, period=$2$

$\sqrt{10} = [1;(6)]$, period=$1$

$\sqrt{11} = [1;(3, 5)]$, period=$2$

$\sqrt{12} = [1;(2,5)]$, period=$2$

$\sqrt{13} = [1;(1,1,1,1,6)]$, period=$5$

Exactly four continued fractions, for $N \le 13$, have an odd period.

How many continued fractions for $N \le 10 000$ have an odd period?
