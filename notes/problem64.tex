\section{Odd Period Square Roots}
\subsection{Problem 64}

All square roots are periodic when written as continued fractions and can be written in the form:
\begin{equation*}
	\sqrt{N} = a_0 + \cfrac{1}{a_1 + \cfrac{1}{a_2 + \cfrac{1}{a_3 + \dots}}}
\end{equation*}

For example, let us consider \( \sqrt{23} \):
\begin{equation*}
	\sqrt{23} = 4 + \sqrt{23} - 4 = 4 + \cfrac{1}{\cfrac{1}{\sqrt{23}-4}} = 4 + \cfrac{1}{1 + \cfrac{\sqrt{23} - 3}{7}}
\end{equation*}

If we continue we would get the following expansion:
\begin{equation*}
	\sqrt{23} = 4 + \cfrac{1}{1 + \cfrac{1}{3 + \cfrac{1}{1 + \cfrac{1}{8 + \dots}}}}
\end{equation*}

The process can be summarised as follows:

\begin{align*}
	a_0 & = 4, \frac{1}{\sqrt{23} - 4}  = \frac{\sqrt{23} + 4}{7}      = 1 + \frac{\sqrt{23} -3}{7} \\
	a_1 & = 1, \frac{7}{\sqrt{23} - 3}  = \frac{7(\sqrt{23} + 3)}{14}  = 3 + \frac{\sqrt{23} -3}{2} \\
	a_2 & = 3, \frac{2}{\sqrt{23} - 3}  = \frac{2(\sqrt{23} + 3)}{14}  = 1 + \frac{\sqrt{23} -4}{7} \\
	a_3 & = 1, \frac{7}{\sqrt{23} - 4}  = \frac{7(\sqrt{23} + 4)}{7}   = 8 + {\sqrt{23} -4}         \\
	a_4 & = 8, \frac{1}{\sqrt{23} - 4}  = \frac{\sqrt{23} + 4}{7}      = 1 + \frac{\sqrt{23} -3}{7} \\
	a_5 & = 1, \frac{7}{\sqrt{23} - 3}  = \frac{7(\sqrt{23} + 3)}{14}  = 3 + \frac{\sqrt{23} -3}{2} \\
	a_6 & = 3, \frac{2}{\sqrt{23} - 3}  = \frac{2(\sqrt{23} + 3)}{14}  = 1 + \frac{\sqrt{23} -4}{7} \\
	a_7 & = 1, \frac{7}{\sqrt{23} - 4}  = \frac{7(\sqrt{23} + 4)}{7}   = 8 + {\sqrt{23} -4}         \\
\end{align*}

It can be seen that the sequence is repeating. For conciseness, we use the notation \( \sqrt{23} = [4;(1,3,1,8)] \), to
indicate that the block (1,3,1,8) repeats indefinitely.

The first ten continued fraction representations of (irrational) square roots are:
\begin{align*}
	\sqrt{2}  & = [1;(2)],         & \text{period} & =1 \\
	\sqrt{3}  & = [1;(1,2)],       & \text{period} & =2 \\
	\sqrt{5}  & = [2;(4)],         & \text{period} & =1 \\
	\sqrt{6}  & = [2;(2,4)],       & \text{period} & =2 \\
	\sqrt{7}  & = [1;(1,1,1,4)],   & \text{period} & =4 \\
	\sqrt{8}  & = [1;(1,4)],       & \text{period} & =2 \\
	\sqrt{10} & = [1;(6)],         & \text{period} & =1 \\
	\sqrt{11} & = [1;(3,5)],       & \text{period} & =2 \\
	\sqrt{12} & = [1;(2,5)],       & \text{period} & =2 \\
	\sqrt{13} & = [1;(1,1,1,1,6)], & \text{period} & =5 \\
\end{align*}

Exactly four continued fractions, for \( N \le 13 \), have an odd period.

How many continued fractions for \( N \le 10 000 \) have an odd period?
