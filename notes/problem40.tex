\section{Champernowne 常数}
\subsection{问题描述}
\begin{tcolorbox}

	Champernowne 常数是通过将正整数序列连接起来构成的一个无限小数。例如:

	\[
		0.12345678910\red{1}112131415161718192021\ldots
	\]

	这个小数包含了所有的正整数数字,依次排列。

	给定 Champernowne 常数的第 \( n \) 位是指小数中第 \( n \) 位的数字。

	求出下列位置上的数字的乘积:

	\[
		d_1 \times d_{10} \times d_{100} \times d_{1000} \times d_{10000} \times d_{100000} \times d_{1000000}
	\]

	其中 \( d_n \) 表示 Champernowne 常数中第 \( n \) 位的数字。
\end{tcolorbox}

\subsection{算法}
\begin{enumerate}
	\item 计算目标所在的区域,将$ n $ 循环减去 \( 9, 90 \times 2, 900 \times 3 \)直到 \( n \) 小于当前数字宽带的数字的总和。
	\item 将 \( n  - 1\) 与当前的数字宽度求余,再加之前的所有数字即得到第 \( n \)位数对应的数 \( P \)
	\item 计算目标在 	\( P \)上的索引号,然后将其取出。
\end{enumerate}

\subsection{答案}
210
