\section{组合数计算}\label{sec:problem53}
\subsection{问题描述}
\begin{tcolorbox}
已知有 10 种从 5 个数 12345 中选出 3 个数的方法:

\[ 
123, 124, 125, 134, 135, 145, 234, 235, 245, 345 
\]

在组合数学中,我们使用以下符号来表示组合数:

\[
\binom{5}{3} = 10
\]

一般来说,组合数公式为:

\[
\binom{n}{r} = \frac{n!}{r!(n - r)!}
\]

其中,\(r \leq n\),且 \(n! = n \times (n-1) \times \cdots \times 1\),且 \(0! = 1\)。

当 \(n = 23\) 时,组合数首次超过 100 万:

\[
\binom{23}{10} = 1144066
\]

问题:在 \( 1 \leq n \leq 100 \) 的范围内,有多少个不一定互不相同的组合数 \( \binom{n}{r} \) 的值大于一百万?

\end{tcolorbox}

\subsection{算法}
这道题当然可以用设计的大数来解决,但是如果采取一些优化算法还是可以避免溢出的。

分子与分母每一次计算时都用最小公倍数来减小。

第二种是用combination的另外一个递推公式:
\begin{equation*}
\binom{n}{r}=\binom{n-1}{r-1}+\binom{n-1}{r}
\end{equation*}
这个方法更高效和通用,也不会有溢出的风险。
\subsection{答案}
4075
