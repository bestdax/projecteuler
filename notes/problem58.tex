\section{螺旋素数}
\subsection{问题描述}
\begin{tcolorbox}
	从数字 $1$ 开始,按照逆时针方向生成一个方形螺旋矩阵。例如,一个边长为 7 的方形螺旋矩阵如下:
	\[
		\begin{matrix}
			\red{37} & 36       & 35      & 34 & 33      & 32       & \red{31} \\
			38       & \red{17} & 16      & 15 & 14      & \red{13} & 30       \\
			39       & 18       & \red{5} & 4  & \red{3} & 12       & 29       \\
			40       & 19       & 6       & 1  & 2       & 11       & 28       \\
			41       & 20       & \red{7} & 8  & 9       & 10       & 27       \\
			42       & 21       & 22      & 23 & 24      & 25       & 26       \\
			\red{43} & 44       & 45      & 46 & 47      & 48       & 49       \\
		\end{matrix}
	\]

	其中,奇数平方数位于右下对角线上。在 7x7 的螺旋矩阵中,沿着对角线的 13 个数中有 8 个是质数,比例约为 $62\%$。

	问题:当螺旋矩阵的边长增加时,对角线上质数的比例首次下降到 $10\%$ 以下时,矩阵的边长是多少?
\end{tcolorbox}

\subsection{算法}
每增加一圈,加长加2,角点的步进等于当前边长-1。

这道题不能用埃氏筛法来解,效率比较低,因为随着边长的增加,角点数之间的间隔越来越大,判断角点数之间的数是效率的浪费。

\subsection{答案}
26241
