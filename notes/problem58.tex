\chapter{Spiral Primes}
Spiral Primes
Starting with 1 and spiralling anticlockwise in the following way, a square spiral with side length 7 is formed.
\begin{center}
	\[
		\begin{matrix}
			{\textcolor{red}{37}} & 36                  & 35                 & 34 & 33                 & 32                  & \textcolor{red}{31} \\
			38                    & \textcolor{red}{17} & 16                 & 15 & 14                 & \textcolor{red}{13} & 30                  \\
			39                    & 18                  & \textcolor{red}{5} & 4  & \textcolor{red}{3} & 12                  & 29                  \\
			40                    & 19                  & 6                  & 1  & 2                  & 11                  & 28                  \\
			41                    & 20                  & \textcolor{red}{7} & 8  & 9                  & 10                  & 27                  \\
			42                    & 21                  & 22                 & 23 & 24                 & 25                  & 26                  \\
			\textcolor{red}{43}   & 44                  & 45                 & 46 & 47                 & 48                  & 49                  \\
		\end{matrix}
	\]
\end{center}
It is interesting to note that the odd squares lie along the bottom right diagonal, but what is more interesting is that
8 out of the 13 numbers lying along both diagonals are prime; that is, a ratio of 8/13 ≈ 62.

If one complete new layer is wrapped around the spiral above, a square spiral with side length 9 will be formed. If this
process is continued, what is the side length of the square spiral for which the ratio of primes along both diagonals
first falls below 10?
