\section{可截素数}
\subsection{问题描述}
\begin{tcolorbox}

可截素数指的是:从左到右和从右到左逐位截去数字后,剩下的所有数字仍然是素数。例如,3797 是一个可截素数:

\begin{itemize}
    \item 从左到右依次截去:3797、797、97、7 都是素数;
    \item 从右到左依次截去:3797、379、37、3 也是素数。
\end{itemize}

只考虑两位数及以上的素数,求所有这样的素数之和(一共只有11个)。

\end{tcolorbox}

\subsection{算法}

依次用 \( 10, 100, \cdots \) 来对要判断的质数 \( n \)取余,并依次判断余数是不是质数。

\subsection{答案}
748317
