\section{有序分数}
\subsection{问题描述}
\begin{tcolorbox}
考虑分数 \( \frac{n}{d} \),其中 \( n \) 和 \( d \) 是正整数。如果 \( n < d \) 且 \( \text{HCF}(n, d) = 1 \),它被称为简化真分数。

如果我们列出 \( d \leq 8 \) 的所有简化真分数,并按大小升序排列,我们得到:
\[ \frac{1}{8}, \frac{1}{7}, \frac{1}{6}, \frac{1}{5}, \frac{1}{4}, \frac{2}{7}, \frac{1}{3}, \frac{3}{8}, \frac{\bf
2}{\bf 5}, \frac{3}{7}, \frac{1}{2}, \frac{4}{7}, \frac{3}{5}, \frac{5}{8}, \frac{2}{3}, \frac{5}{7}, \frac{3}{4}, \frac{4}{5}, \frac{5}{6}, \frac{6}{7}, \frac{7}{8} \]

可以看出 \( \frac{2}{5} \) 是 \( \frac{3}{7} \) 左边的分数。

通过列出 \( d \leq 1000000 \) 的所有简化真分数,并按大小升序排列,找到 \( \frac{3}{7} \) 左边的分数的分子。

\end{tcolorbox}

\subsection{算法}


\subsection{答案}
