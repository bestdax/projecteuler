\section{欧拉函数排列}
\subsection{问题描述}
\begin{tcolorbox}
\begin{itemize}
    \item 欧拉的欧拉函数 \(\phi(n)\)(有时也称为phi函数),用于确定小于或等于 \(n\) 的正整数中有多少个与 \(n\) 互质的数。例如,因为 \(1, 2, 4, 5, 7,\) 和 \(8\) 都小于九且与九互质,所以 \(\phi(9) = 6\)。
    \item 数字 \(1\) 被认为是与每个正整数互质的,所以 \(\phi(1) = 1\)。
    \item 有趣的是,\(\phi(87109) = 79180\),并且可以看出 \(87109\) 是 \(79180\) 的一个排列。
    \item 找出 \(n\) 的值,使得 \(1 < n < 10^7\),\(\phi(n)\) 是 \(n\) 的一个排列,并且比率 \(n / \phi(n)\) 产生最小值。
\end{itemize}
\end{tcolorbox}

\subsection{算法}
用问题69中的欧拉函数筛法得到 \( \varphi(n) \)的数组,然后判定 \( \varphi(n) \)是不是 \( n \)的排列数。

\subsection{答案}
8319823
