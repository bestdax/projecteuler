\section{非盈数之和}
\subsection{问题描述}
\begin{tcolorbox}
	如果一个数的真因数之和大于该数本身,则称该数为\textbf{盈数}。例如,$12$ 是最小的盈数,它的真因数之和为:
	\[
		1 + 2 + 3 + 4 + 6 = 16
	\]
	大于12。找出所有不能写成两个盈数之和的正整数,并求它们的总和。

	已知:所有大于28123的数字都可以写成两个盈数之和,因此我们只需考虑小于等于28123的数字。

\end{tcolorbox}

\subsection{算法}
\begin{enumerate}
	\item \textbf{判断盈数}:编写函数判断某个数是否为盈数。
	\item \textbf{生成盈数}:找出所有小于等于28123的盈数。
	\item \textbf{判断是否为两个盈数之和}:通过分解一个数字为两个部分,分别计算是否为盈数。
	\item \textbf{求和}:遍历所有小于等于28123的数字,计算那些不能表示为两个盈数之和的数字总和。
\end{enumerate}

\subsection{答案}
4179871
