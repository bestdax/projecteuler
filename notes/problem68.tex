\section{魔法五边形环}
\subsection{问题描述}
\begin{tcolorbox}[breakable]
	考虑以下“魔法”三角形环,填充数字1到6,每条线之和为9。

	按照顺时针方向,从数值最小的外部节点开始(例如本例中的 \( 4,3,2
	\)),每个解都可以被唯一描述。例如,上述解可以由以下集合描述: \( 4,3,2; 6,2,1; 5,1,3 \)。

	有可能完成环的总和有四种不同的值: \( 9、10、11 \)和 \( 12 \)。总共有八个解。

	\begin{center}
		\begin{tikzpicture}[every node/.style={shape=circle, draw}]
			\coordinate(1) at (0,0);
			\coordinate(2) at (2,0);
			\coordinate(6) at (4,0);
			\coordinate(3) at (60:2);
			\coordinate(4) at ($(2)!2!(3)$);
			\coordinate(5) at ($(3)!2!(1)$);
			\node(n1) at (1) {1};
			\node(n2) at (2) {2};
			\node(n3) at (3) {3};
			\node(n4) at (4) {4};
			\node(n5) at (5) {5};
			\node(n6) at (6) {6};
			\draw (n1) -- (n2);
			\draw (n2) -- (n3);
			\draw (n3) -- (n4);
			\draw (n2) -- (n6);
			\draw (n3) -- (n1);
			\draw (n1) -- (n5);
		\end{tikzpicture}
	\end{center}

	\begin{center}
		\begin{tabular}{cc}
			\textbf{Total} & \textbf{Solution Set} \\
			9              & 4,2,3; 5,3,1; 6,1,2   \\
			9              & 4,3,2; 6,2,1; 5,1,3   \\
			10             & 2,3,5; 4,5,1; 6,1,3   \\
			10             & 2,5,3; 6,3,1; 4,1,5   \\
			11             & 1,4,6; 3,6,2; 5,2,4   \\
			11             & 1,6,4; 5,4,2; 3,2,6   \\
			12             & 1,5,6; 2,6,4; 3,4,5   \\
			12             & 1,6,5; 3,5,4; 2,4,6   \\
		\end{tabular}
	\end{center}

通过连接每个组,可以形成9位数字的字符串;对于一个三角形环,最大的字符串是$432621513$。

使用数字1到10,并且根据排列,可以形成16位和17位的字符串。一个“魔法”五边形环的最大16位数字字符串是什么?

	\begin{center}
		\begin{tikzpicture}[every node/.style={shape=circle, draw, minimum height=0.7cm}]
			\coordinate(1) at (0,0);
			\coordinate(2) at (2,0);
			\coordinate(3) at (4,0);
			\coordinate(4) at ([xshift=2cm]72:2);
			\coordinate(5) at ($(2)!2!(4)$);
			\coordinate(6) at ($(4) + (144:2)$);
			\coordinate(7) at ($(4)!2!(6)$);
			\coordinate(8) at ($(6) + (216:2)$);
			\coordinate(9) at ($(6)!2!(8)$);
			\coordinate(10) at ($(8)!2!(1)$);
			\node(n1) at (1) {};
			\node(n2) at (2) {};
			\node(n3) at (3) {};
			\node(n4) at (4) {};
			\node(n5) at (5) {};
			\node(n6) at (6) {};
			\node(n7) at (7) {};
			\node(n8) at (8) {};
			\node(n9) at (9) {};
			\node(n10) at (10) {};
			\draw (n1) -- (n2);
			\draw (n2) -- (n3);
			\draw (n2) -- (n4);
			\draw (n4) -- (n5);
			\draw (n4) -- (n6);
			\draw (n6) -- (n7);
			\draw (n6) -- (n8);
			\draw (n8) -- (n9);
			\draw (n8) -- (n1);
			\draw (n10) -- (n1);
		\end{tikzpicture}
	\end{center}
\end{tcolorbox}

\subsection{算法}
这道题做起来不难,但是却有点搞。所以如果10在内圈,则肯定会被使用2次,最终会形成17位的数字,所以只能是在外圈。

可以尝试将1--5放在内圈,6--10在外圈,利用排列内圈的方式计算各个分支的和是不是等于14。可以打印出结果,结合图形手动计算出结果后再用代码拼出最后的答案。

\subsection{答案}
6531031914842725
\begin{center}
	\begin{tikzpicture}[every node/.style={shape=circle, draw, minimum height=0.7cm}]
		\coordinate(1) at (0,0);
		\coordinate(2) at (2,0);
		\coordinate(3) at (4,0);
		\coordinate(4) at ([xshift=2cm]72:2);
		\coordinate(5) at ($(2)!2!(4)$);
		\coordinate(6) at ($(4) + (144:2)$);
		\coordinate(7) at ($(4)!2!(6)$);
		\coordinate(8) at ($(6) + (216:2)$);
		\coordinate(9) at ($(6)!2!(8)$);
		\coordinate(10) at ($(8)!2!(1)$);
		\node(n1) at (1) {3};
		\node(n2) at (2) {5};
		\node(n3) at (3) {6};
		\node(n4) at (4) {2};
		\node(n5) at (5) {7};
		\node(n6) at (6) {4};
		\node(n7) at (7) {8};
		\node(n8) at (8) {1};
		\node(n9) at (9) {9};
		\node(n10) at (10) {10};
		\draw (n1) -- (n2);
		\draw (n2) -- (n3);
		\draw (n2) -- (n4);
		\draw (n4) -- (n5);
		\draw (n4) -- (n6);
		\draw (n6) -- (n7);
		\draw (n6) -- (n8);
		\draw (n8) -- (n9);
		\draw (n8) -- (n1);
		\draw (n10) -- (n1);
	\end{tikzpicture}
\end{center}

