\section{全数字乘积}\label{sec:problem32}
\subsection{问题描述}
\begin{tcolorbox}
	我们称一个 $n$ 位数字为“全数字”的,如果它恰好使用了从 1 到 $n$ 的所有数字一次;例如,5 位数字 $15234$ 就是 1 到 5 的全数字。

	数字 $7254$ 是不寻常的,因为等式 $39 \times 186 = 7254$ 中的乘数、乘数和乘积都是 1 到 9 的全数字。

	找出所有被乘数/乘数/乘积等式可以写成 1 到 9 的全数字的所有乘积的和。

	提示:有些乘积可以通过不止一种方式获得,因此确保在你的总和中只包含一次。
\end{tcolorbox}

\subsection{算法}
将1-9这几个数字组成的字符串用\texttt{next\_permutation},然后对字符串进行分割。一共只有两种方式, \( (1,4,4)\) 和\( (2,3,4)
\),分割之后检查乘法是否成立。因为有可能会有重复的乘积,所以用\texttt{unordered\_set}来存储结果。

\subsection{答案}
$45228$
