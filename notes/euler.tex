%!TEX program = lualatex
\documentclass{ctexart}
\usepackage{amsmath, amssymb, amsthm, mathtools}
\usepackage[a4paper, margin=2cm]{geometry}
\usepackage{indentfirst}

\hfuzz=2pt
%%%%%%%%%%%%%%%%%%%%%%%%%%%%%%%%%%%%%%%%%%%%%%%%%%%%%%%%%%%%%%%%%%%%%%%%%%%%%%%
% filter warnings
\usepackage{silence}
\WarningFilter{latexfont}{Font shape}
\WarningFilter{latexfont}{Some font}
%%%%%%%%%%%%%%%%%%%%%%%%%%%%%%%%%%%%%%%%%%%%%%%%%%%%%%%%%%%%%%%%%%%%%%%%%%%%%%%
\usepackage[outputdir=build]{minted}
\usepackage[all]{tcolorbox}
\newtcblisting{cpp}{listing engine=minted, minted language=cpp, minted style=autumn, listing only}
\newtcolorbox{tablebox}[1]{width=\textwidth, right=0mm, left=0mm, top=0mm, bottom=0mm, boxsep=0mm, before title=\strut, halign
title=center,  valign=center, title=#1, halign=center, colframe=green!70!blue!50}

% tikz settings
\usepackage{tikz}
\usetikzlibrary{shapes.geometric, arrows, shapes.misc, calc}
\usepackage[hypcap=false]{caption}
\tikzstyle{startstop} = [rectangle, rounded corners, minimum width=3cm, minimum height=1cm,text centered, draw=black, fill=red!30]
\tikzstyle{operation} = [align=center, rectangle, minimum width=3cm, minimum height=1cm,text centered, draw=black, fill=orange!30]
\tikzstyle{decision} = [diamond, minimum width=4cm, minimum height=1cm, align=center, text centered, draw=black, fill=green!30]
\tikzstyle{arrow} = [thick,->,>=stealth]
\tikzstyle{io} = [trapezium, trapezium left angle=70, trapezium right angle=110, minimum width=3cm, minimum height=1cm, text centered, draw=black, fill=blue!30]


% for tables
\usepackage{array}
\usepackage{tabularx}
\usepackage{tabulary}
\usepackage{colortbl}
\usepackage{multirow}
\usepackage{hyperref}
\usepackage{enumitem}
\usepackage{csquotes}
%%%%%%%%%%%%%%%%%%%%%%%%%%%%%%%%%%%%%%%%%%%%%%%%%%%%%%%%%%%%%%%%%%%%%%%%%%%%%%%
\newtheorem{definition}{定义}
\newtheorem{theorem}{定理}
\newtheorem{example}{例}
%%%%%%%%%%%%%%%%%%%%%%%%%%%%%%%%%%%%%%%%%%%%%%%%%%%%%%%%%%%%%%%%%%%%%%%%%%%%%%%
\newcommand{\eqrefl}[1]{
\foreach \eq in {#1}
{
  \eqref{\eq},
}
}

%%%%%%%%%%%%%%%%%%%%%%%%%%%%%%%%%%%%%%%%%%%%%%%%%%%%%%%%%%%%%%%%%%%%%%%%%%%%%%%
\author{dax}
\title{Project Euler Tutorial}
\date{2023/05/24}

\begin{document}
\maketitle
\section{3或5的倍数}\label{sec:problem01}
\subsection{Problem Description}
\begin{tcolorbox}
	如果我们列出所有小于10的自然数,这些数是3或5的倍数,我们会得到3、5、6和9。它们的和是23。

	请找出所有小于1000的3或5的倍数的和。

	\href{https://projecteuler.net/problem=1}{原问题链接}
\end{tcolorbox}

\subsection{算法}
\begin{algorithm}
	\caption{找到3或5的倍数的和}
	\begin{algorithmic}
		\State 初始化 sum $\gets 0$
		\For{$i = 1$ \to $N-1$}
		\If{$i \mod 3 = 0$ \textbf{or} $i \mod 5 = 0$}
		\State sum $\gets$ sum + $i$
		\EndIf
		\EndFor
		\Return sum
	\end{algorithmic}
\end{algorithm}
这里面要注意题目中要求的是小于上限。

\subsection{答案}
233168

\chapter{Even Fibonacci numbers}
\section{Description}
Each new term in the Fibonacci sequence is generated by adding the previous two terms. By starting with 1 and 2, the first 10 terms will be:
\[1, 2, 3, 5, 8, 13, 21, 34, 55, 89, \dots\]

By considering the terms in the Fibonacci sequence whose values do not exceed four million, find the sum of the even-valued terms.

\section{Flow Chart}
\begin{center}
	\begin{tikzpicture}[node distance=2cm]
		\node (start) [startstop] {开始};
		\node (input) [io, right of=start, xshift=4cm] {输入一个上限值cap};
		\node (op1) [operation, below of=input] {a = 0; \\
			b = 1\\
			sum = 0;};
		\node (op2) [operation, below of=op1] {c = a + b;\\
			a = b;\\
			b = c;};
		\node (dec1) [decision, below of=op2, yshift=-.5cm] {c \% 2 == 0};
		\node (op3) [operation, below of=dec1, yshift=-.3cm] {sum += c;};
		\node (dec2) [decision, below of=op3] {c < cap};
		\node (output) [io, below of=dec2] {输出答案};
		\node (end) [startstop, left of=output, xshift=-4cm] {结束};

		\draw [arrow] (start) -- (input);
		\draw [arrow] (input) -- (op1);
		\draw [arrow] (op1) -- (op2);
		\draw [arrow] (op2) -- (dec1);
		\draw [arrow] (dec1) -- node[right]{Yes} (op3);
		\draw [arrow] (op3) --  (dec2);
		\draw [arrow] (dec2) -- node[right]{No} (output);
		\draw [arrow] (dec2.west) -| node[above=3cm, xshift=-.5cm] {Yes}([xshift=-2cm]op2.west) --  (op2.west);
		\draw [arrow] (output) --  (end);
	\end{tikzpicture}
	\captionof{figure}{Sum Of Even Fibonacci}
\end{center}

\section{Codes}
\begin{cpp}
	std::vector<int> Solution::fibs(int cap)
	{
		std::vector<int> fb;
		int a = 0, b = 1, c = 1;

		while(c <= cap)
		{
			fb.push_back(c);
			a = b;
			b = c;
			c = a + b;
		}

		return fb;
	}

	long Solution::sum_of_even_fibs(int cap)
	{
		long sum{};
		std::vector<int> fb = fibs(cap);

		for(auto item : fb)
		{
			if(item % 2 == 0) sum += item;
		}

		return sum;
	}
\end{cpp}

\section{扑克游戏}\label{sec:problem54}
\subsection{问题描述}
\begin{tcolorbox}[breakable]
	在扑克牌游戏扑克(Poker)中,一手牌由五张牌组成,并且按照以下方式从低到高排序:

	\begin{itemize}
		\item \textbf{高牌(High Card)}:最大的一张牌。
		\item \textbf{一对(One Pair)}:两张相同点数的牌。
		\item \textbf{两对(Two Pairs)}:两个不同点数的对子。
		\item \textbf{三条(Three of a Kind)}:三张相同点数的牌。
		\item \textbf{顺子(Straight)}:五张连续点数的牌。
		\item \textbf{同花(Flush)}:五张同花色的牌。
		\item \textbf{葫芦(Full House)}:三张相同点数的牌加一对。
		\item \textbf{四条(Four of a Kind)}:四张相同点数的牌。
		\item \textbf{同花顺(Straight Flush)}:五张同花色且连续点数的牌。
		\item \textbf{皇家同花顺(Royal Flush)}:10、J、Q、K、A同花色的牌。
	\end{itemize}

	牌的点数按照以下顺序从低到高排序:2, 3, 4, 5, 6, 7, 8, 9, 10, J, Q, K, A。

	如果两位玩家的手牌是同一种类型,则比较最大的一张牌。举例来说,成对的八(Pair of Eights)比成对的五(Pair of Fives)要大。如果两手牌中的最大牌相同,则比较次高的牌,依次类推。

	考虑以下两位玩家的五手牌对决:

	\[
		\begin{array}{|c|c|c|c|}
			\hline
			\text{手牌} & \text{玩家1}          & \text{玩家2}          & \text{胜者}   \\
			\hline
			1           & \texttt{5H 5C 6S 7S KD} & \texttt{2C 3S 8S 8D TD} & \text{玩家2 } \\
			2           & \texttt{5D 8C 9S JS AC} & \texttt{2C 5C 7D 8S QH} & \text{玩家1 } \\
			3           & \texttt{2D 9C AS AH AC} & \texttt{3D 6D 7D TD QD} & \text{玩家2 } \\
			4           & \texttt{4D 6S 9H QH QC} & \texttt{3D 6D 7H QD QS} & \text{玩家1 } \\
			5           & \texttt{2H 2D 4C 4D 4S} & \texttt{3C 3D 3S 9S 9D} & \text{玩家1 } \\
			\hline
		\end{array}
	\]

    文件 \href{https://projecteuler.net/resources/documents/0054_poker.txt}{poker.txt} 包含1000局随机生成的牌局,每局包含两位玩家的十张牌(以空格分隔),前五张是玩家1的手牌,后五张是玩家2的手牌。可以假设所有的手牌均合法(无无效字符或重复卡牌),每位玩家的手牌无顺序,并且在每局游戏中有明确的胜者。

	问题:玩家1赢得了多少局?

\end{tcolorbox}

\subsection{算法}
设计一个\texttt{PokerHand}类,比较大小用 \texttt{<=>}来判定大小:
\begin{enumerate}
    \item 如果手牌的种类不同,种类级别高的赢
    \item 如果手牌的种类相同,则依次比较
\end{enumerate}
要注意题目中不分花色的大小,另外皇家同花顺与同花顺没有必要作区分。

\subsection{答案}
376

\chapter{Lychrel Numbers}

\section{Problem Description}
If we take $47$, reverse and add,$47 + 74 = 121$, which is palindromic.

Not all numbers produce palindromes so quickly. For example,
\begin{align*}
	349 + 943   & = 1292 \\
	1292 + 2921 & = 4213 \\
	4213 + 3124 & = 7337
\end{align*}

That is, $349$ took three iterations to arrive at a palindrome.

Although no one has proved it yet, it is thought that some numbers, like $196$ 
, never produce a palindrome. A number that never forms a palindrome through the reverse and add process is called a
Lychrel number. Due to the theoretical nature of these numbers, and for the purpose of this problem, we shall assume
that a number is Lychrel until proven otherwise. In addition you are given that for every number below ten-thousand, it
will either (i) become a palindrome in less than fifty iterations, or, (ii) no one, with all the computing power that
exists, has managed so far to map it to a palindrome. In fact, $10677$
is the first number to be shown to require over fifty iterations before producing a palindrome:4668731596684224866951378664
($53$ iterations, $28$ digits).

Surprisingly, there are palindromic numbers that are themselves Lychrel numbers; the first example is $4994$.

How many Lychrel numbers are there below ten-thousand?

NOTE: Wording was modified slightly on 24 April 2007 to emphasise the theoretical nature of Lychrel numbers.

\section{Flow Chart}

\begin{tikzpicture}[node distance=2cm]
  \node [startstop] (start) {开始};
  \node [io, right of=start, xshift=4cm, text width=4cm] (io1) {输入一个考察的上限\\ 这里是10000};
  \node [operation, text width=2cm, below of=io1] (op1) {count = 0;\\ i = 10;};
  \node [decision, below of=op1, text width=2cm, yshift=-1cm] (dec1) {$i < 上限$};
  \node [decision, below of=dec1, text width=2cm, yshift=-2cm] (dec2) {$i$ is lychrel number?};
  \node [operation, below of=dec2, yshift=-1cm] (op2) {++count};
  \node [operation, right of=dec2, xshift= 4cm] (op3) {++i};
  \node [io, below of=op2] (io2) {输出count};
  \node [startstop, right of=io2, xshift=4cm] (stop) {结束};

  \draw[arrow] (start) -- (io1);
  \draw[arrow] (io1) -- (op1);
  \draw[arrow] (op1) -- (dec1);
  \draw[arrow] (dec1) -- node[right]{yes} (dec2);
  \draw[arrow] (dec1) -| node[right=.5cm, below=4cm]{no}([xshift=-2cm]io2.west) -- (io2);
  \draw[arrow] (dec2) -- node[right]{yes} (op2);
  \draw[arrow] (dec2) -- node[above]{no}(op3);
  \draw[arrow] (op3) |- (dec1);
  \draw[arrow] (io2) -- (stop);
\end{tikzpicture}

\section{幂的数字和}
\subsection{问题描述}
\begin{tcolorbox}
一个 googol($10^{100}$)是一个非常大的数字:它由一百个零组成。类似地,$100^{100}$ 也是一个几乎难以想象的庞大数字,由两百个零组成。然而,它们的数字和都只有 1。

现在考虑形式为 $a^b$ 的自然数,其中 $a, b < 100$。求出这些自然数中数字和的最大值。
\end{tcolorbox}

\subsection{算法}
这又是一道大数题,用BigUInt来解决。

\subsection{答案}
972

\section{方根的渐进分数}
\subsection{问题描述}
\begin{tcolorbox}
	可以证明,$\sqrt{2}$ 可以表示为一个无限的连分数:
	\[
		\sqrt{2} = 1 + \cfrac{1}{2 + \cfrac{1}{2 + \cfrac{1}{2 + \dots}}}
	\]

	通过展开前四次迭代,我们得到:
	\[
		1 + \cfrac{1}{2} = \cfrac{3}{2} = 1.5
	\]
	\[
		1 + \cfrac{1}{2 + \cfrac{1}{2}} = \cfrac{7}{5} = 1.4
	\]
	\[
		1 + \cfrac{1}{2 + \cfrac{1}{2 + \cfrac{1}{2}}} = \cfrac{17}{12} = 1.41666 \dots
	\]
	\[
		1 + \cfrac{1}{2 + \cfrac{1}{2 + \cfrac{1}{2 + \cfrac{1}{2}}}} = \cfrac{41}{29} = 1.41379 \dots
	\]

	在第八次展开时,$\cfrac{1393}{985}$ 是第一个分子位数超过分母位数的例子。

	问题:在前 1000 次展开中,有多少个分数的分子的位数多于分母的位数?
\end{tcolorbox}

\subsection{算法}
这道题涉及到连分数的知识, $ \sqrt{2} $的连分数可以表示成 $ \langle 1, \overline{2} \rangle $, 分子分母的递推公式如下:
\begin{align*}
	P_n & = a_nP_{n-1} + P_{n-2} \\
	Q_n & = a_nQ_{n-1} + Q_{n-2} \\
\end{align*}
其中 \( P_{-1} = 1, P_{-2} = 0, Q_{-1} = 0, Q_{-2} = 1 \)

当然这道题还是要用到大整数,1000次展开将是一个很大的数。

\subsection{答案}
153

\section{螺旋素数}
\subsection{问题描述}
\begin{tcolorbox}
	从数字 $1$ 开始,按照逆时针方向生成一个方形螺旋矩阵。例如,一个边长为 7 的方形螺旋矩阵如下:
	\[
		\begin{matrix}
			\red{37} & 36       & 35      & 34 & 33      & 32       & \red{31} \\
			38       & \red{17} & 16      & 15 & 14      & \red{13} & 30       \\
			39       & 18       & \red{5} & 4  & \red{3} & 12       & 29       \\
			40       & 19       & 6       & 1  & 2       & 11       & 28       \\
			41       & 20       & \red{7} & 8  & 9       & 10       & 27       \\
			42       & 21       & 22      & 23 & 24      & 25       & 26       \\
			\red{43} & 44       & 45      & 46 & 47      & 48       & 49       \\
		\end{matrix}
	\]

	其中,奇数平方数位于右下对角线上。在 7x7 的螺旋矩阵中,沿着对角线的 13 个数中有 8 个是质数,比例约为 $62\%$。

	问题:当螺旋矩阵的边长增加时,对角线上质数的比例首次下降到 $10\%$ 以下时,矩阵的边长是多少?
\end{tcolorbox}

\subsection{算法}
每增加一圈,加长加2,角点的步进等于当前边长-1。

这道题不能用埃氏筛法来解,效率比较低,因为随着边长的增加,角点数之间的间隔越来越大,判断角点数之间的数是效率的浪费。

\subsection{答案}
26241

\section{XOR 解密}\label{sec:problem59}
\subsection{问题描述}
\begin{tcolorbox}

	每个计算机上的字符都被分配了一个唯一的代码,优选标准是 ASCII(美国标准信息交换代码)。例如,大写字母 A = 65,星号 (*) = 42,小写字母 k = 107。

	一种现代加密方法是将文本文件转换为 ASCII,然后将每个字节与来自密钥的给定值进行 XOR 操作。使用相同的加密密钥对密文进行操作,可以恢复明文。

	你的任务是解密一段由三个小写字母组成的加密密钥生成的消息,并找出原始文本中 ASCII 值的总和。

	请\href{https://projecteuler.net/resources/documents/0059_cipher.txt}{点击}下载文本。
\end{tcolorbox}

\subsection{算法}
用异或生成文本,然后在文本中搜索\texttt{" the "},要包括两边的空格。

\subsection{答案}
129448

\section{素数对集}
\subsection{问题描述}
\begin{tcolorbox}
素数 $3$, $7$, $109$, 和 $673$ 是非常显著的。取任意两个素数并按任意顺序连接,结果总是一个素数。例如,取 $7$ 和 $109$,得到的 $7109$ 和 $1097$ 都是素数。这四个素数的和 $792$ 代表了具有此属性的四个素数集合的最小和。

找到一个五个素数的集合,使得任何两个素数连接后产生的结果仍为素数,并且该集合的和最小。
\end{tcolorbox}

\subsection{算法}
可以先用欧拉筛法得到一个素数数组,然后首先找到满足条件长度为2的素数组合,然后遍历素数组合与素数,逐步增加长度。

用广度优先搜索算法可以提升搜索效率。
\subsection{答案}
26033

\chapter{Cyclical Figurate Numbers}
Triangle, square, pentagonal, hexagonal, heptagonal, and octagonal numbers are all figurate (polygonal) numbers and are
generated by the following formulae:

\begin{align*}
	\text{Triangle}\quad   & P_{3,n} = n(n + 1)/2  & 1,3,6,10,15,\dots  \\
	\text{Square}\quad     & P_{4,n} = n^2         & 1,4,9,16,25,\dots  \\
	\text{Pentagonal}\quad & P_{5,n} = n(3n - 1)/2 & 1,5,12,22,35,\dots \\
	\text{Hexagonal}\quad  & P_{6,n} = n(2n - 1)   & 1,6,15,28,45,\dots \\
	\text{Heptagonal}\quad & P_{7,n} = n(5n -3)/2  & 1,7,18,34,55,\dots \\
	\text{Octagonal}\quad  & P_{8,n} = n(3n -2)    & 1,8,21,40,65,\dots
\end{align*}

The ordered set of three $4$-digit numbers: $8128$, $2882$, $8281$, has three interesting properties.

\begin{enumerate}
	\item The set is cyclic, in that the last two digits of each number is the first two digits of the next number
		(including the last number with the first).
	\item Each polygonal type: triangle ($P_{3,127} = 8128$), square ($P_{4,91} = 8281$), and pentagonal ($P_{5,44} =
		2882$), is represented by a different number in the set.
	\item	This is the only set of $4$-digit numbers with this property.
\end{enumerate}
Find the sum of the only ordered set of six cyclic $4$-digit numbers for which each polygonal type: triangle, square,
pentagonal, hexagonal, heptagonal, and octagonal, is represented by a different number in the set.

\section{立方数排列}
\subsection{问题描述}
\begin{tcolorbox}
数字 \(41063625\) (\(345^3\)) 可以排列成两个其他的立方数:\(56623104\) (\(384^3\)) 和 \(66430125\) (\(405^3\))。实际上,\(41063625\) 是最小的立方数,它的数字排列可以产生正好三个也是立方数的排列。

找出最小的立方数,它的数字排列可以产生正好五个立方数的排列。
\end{tcolorbox}

\subsection{算法}


\subsection{答案}

\section{幂次数字计数}
\subsection{问题描述}
\begin{tcolorbox}
5位数字的数 \( 16807 = 7^5 \) 也是一个五次方。同样地,9位数字的数 \( 134217728 = 8^9 \) 是一个九次方。

有多少个 \( n \) 位正整数同时也是第 \( n \) 次方?
\end{tcolorbox}

\subsection{算法}


\subsection{答案}

\section{title}
\subsection{Problem Description}
\begin{tcolorbox}

\end{tcolorbox}

\section[\lower{e}的连分数]{$e$的连分数}
\subsection{问题描述}
\begin{tcolorbox}[breakable]

	2 的平方根可以表示为一个无限连分数。

	\[ \sqrt{2} = 1 + \dfrac{1}{2 + \dfrac{1}{2 + \dfrac{1}{2 + \cdots}}} \]

	这个无限连分数可以写作:

	\[ \sqrt{2} = [1; (2)] \]

	其中 \((2)\) 表示 2 无限循环。类似地,\(\sqrt{23} = [4; (1, 3, 1, 8)]\)。

	结果表明,连分数的部分值序列提供了平方根的最佳有理数近似。让我们考虑 \(\sqrt{2}\) 的连分数近似值。

	\[ 1 + \dfrac{1}{2} = \dfrac{3}{2} \]

	\[ 1 + \dfrac{1}{2 + \dfrac{1}{2}} = \dfrac{7}{5} \]

	\[ 1 + \dfrac{1}{2 + \dfrac{1}{2 + \dfrac{1}{2}}} = \dfrac{17}{12} \]

	\[ 1 + \dfrac{1}{2 + \dfrac{1}{2 + \dfrac{1}{2 + \dfrac{1}{2}}}} = \dfrac{41}{29} \]

	因此,\(\sqrt{2}\) 的前十个连分数近似值序列是:

	1,
	\[ \dfrac{3}{2}, \dfrac{7}{5}, \dfrac{17}{12}, \dfrac{41}{29}, \dfrac{99}{70}, \dfrac{239}{169}, \dfrac{577}{408}, \dfrac{1393}{985}, \dfrac{3363}{2378}, \ldots \]

	最令人惊讶的是,重要的数学常数 \( e \) 可以表示为:

	\[ e = [2; 1, 2, 1, 1, 4, 1, 1, 6, 1, \ldots, 1, 2k, 1, \ldots] \]

	\( e \) 的前十个连分数近似值序列是:

	2,
	\[ \dfrac{3}{1}, \dfrac{8}{3}, \dfrac{11}{4}, \dfrac{19}{7}, \dfrac{87}{32}, \dfrac{106}{39}, \dfrac{193}{71}, \dfrac{1264}{465}, \dfrac{1457}{536}, \ldots \]

	第 10 个连分数的分子的数字之和是 \(1 + 4 + 5 + 7 = 17\)。

	找到 \( e \) 的连分数的第 100 个连分数的分子的数字之和。
\end{tcolorbox}

\subsection{算法}

\subsection{答案}

\section{丢番图方程}
\subsection{问题描述}
\begin{tcolorbox}
	考虑形式为
	\[ x^2 - D y^2 = 1 \]
	的二次丢番图方程。

	例如,当 \( D = 13 \) 时,\( x \) 的最小解为 \( 649^2 - 13 \times 180^2 = 1 \)。

	可以假设当 \( D \) 是平方数时,不存在正整数解。

	通过寻找 \( D = \{2, 3, 5, 6, 7\} \) 时 \( x \) 的最小解,我们得到以下结果:

	\begin{align*}
		3^2 - 2 \times 2^2       & = 1, \\
		2^2 - 3 \times 1^2       & = 1, \\
		\red{9}^2 - 5 \times 4^2 & = 1, \\
		5^2 - 6 \times 2^2       & = 1, \\
		8^2 - 7 \times 3^2       & = 1.
	\end{align*}

	因此,考虑 \( D \leq 7 \) 时 \( x \) 的最小解,当 \( D = 5 \) 时得到 \( x \) 的最大值。

	找出 \( D \leq 1000 \) 在 \( x \) 的最小解中,使得 \( x \) 的最大值对应的 \( D \) 的值。

\end{tcolorbox}

\subsection{算法}
首先用问题64中的方法生成连分数,然后用连分数渐近分数的递推公式解决这个问题:
\begin{align*}
	P_n     & = a_nP_{n-1} + P_{n-2} \\
	Q_n     & = a_nQ_{n-1} + Q_{n-2} \\
	P_{n-1} & = 1, P_{n-2} = 0       \\
	Q_{n-1} & = 0, Q_{n-2} = 1
\end{align*}

\subsection{答案}
661

\end{document}
