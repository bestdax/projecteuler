%!TEX program = lualatex
\documentclass{ctexbook}
\usepackage{amsmath}
\usepackage[a4paper, margin=2cm]{geometry}
\usepackage{indentfirst}

\usepackage[outputdir=build]{minted}
\usepackage[all]{tcolorbox}
\newtcblisting{cpp}{listing engine=minted, minted language=cpp, minted style=autumn, listing only}
\newtcolorbox{tablebox}[1]{width=\textwidth, right=0mm, left=0mm, top=0mm, bottom=0mm, boxsep=0mm, before title=\strut, halign
title=center,  valign=center, title=#1, halign=center}

% tikz settings
\usepackage{tikz}
\usetikzlibrary{shapes.geometric, arrows, shapes.misc, calc}
\usepackage[hypcap=false]{caption}
\tikzstyle{startstop} = [rectangle, rounded corners, minimum width=3cm, minimum height=1cm,text centered, draw=black, fill=red!30]
\tikzstyle{operation} = [align=center, rectangle, minimum width=3cm, minimum height=1cm,text centered, draw=black, fill=orange!30]
\tikzstyle{decision} = [diamond, minimum width=4cm, minimum height=1cm, align=center, text centered, draw=black, fill=green!30]
\tikzstyle{arrow} = [thick,->,>=stealth]
\tikzstyle{io} = [trapezium, trapezium left angle=70, trapezium right angle=110, minimum width=3cm, minimum height=1cm, text centered, draw=black, fill=blue!30]

\usepackage{fontspec}
\setmainfont[SlantedFont=PingFang SC, BoldFont=PingFang SC Medium, ItalicFont=PingFang SC thin]{PingFang SC}
\setCJKmainfont[SlantedFont=PingFang SC, BoldFont=PingFang SC Medium, ItalicFont=PingFang SC Ultralight]{PingFang SC}
\setsansfont{PingFang SC}
\setmonofont{Hack Nerd Font Mono}

\usepackage{amssymb}

% for tables
\usepackage{array}
\usepackage{tabularx}
\usepackage{tabulary}
\usepackage{colortbl}
\rowcolors{0}{green!20}{white}
\usepackage{multirow}
\setlength{\tabcolsep}{12pt}
\renewcommand{\arraystretch}{1}
\usepackage{hyperref}



\author{dax}
\title{Project Euler Tutorial}
\date{2023/05/24}

\begin{document}
\maketitle
\chapter{Multiples of 3 or 5}
\section{Description}
If we list all the natural numbers below 10 that are multiples of 3 or 5, we get 3, 5, 6 and 9. The sum of these multiples is 23.

Find the sum of all the multiples of 3 or 5 below 1000.
试试看中文能不能显示?
\section{Flow Chart}

\begin{center}
	\begin{tikzpicture}[node distance=2cm]
		\node (start) [startstop] {开始};
		\node (in1) [io, right of=start, xshift=3cm] {输入一个上限cap};
		\node (pro1) [operation, right of=in1, xshift=3cm] {n = 1\\
			sum = 0};
		\node (dec1) [decision, below of=pro1, yshift=-1cm] {n能被3\\
			或者5整除};
		\node (pro2) [yshift=-1cm, operation, below of = dec1] {sum += n;\\
			++n;};
		\node (dec2) [decision, below of=pro2, yshift=-.5cm] {n < cap};
		\node (out1) [io, left of=dec2, xshift=-3cm] {输出数据};
		\node (stop) [startstop, left of=out1, xshift=-3cm] {终止};

		\draw [arrow] (start) -- (in1);
		\draw [arrow] (in1) -- (pro1);
		\draw [arrow] (pro1) -- (dec1);
		\draw [arrow] (dec1) -- node[anchor=east] {Yes}(pro2);
		\draw [arrow] (pro2) --  (dec2);
		\draw [arrow] (dec2.west) -- node[above] {No} (out1);
		\draw [arrow] (dec2.east) -| ([xshift=2cm]dec1.east) -- node[anchor=south east, text centered] {Yes} (dec1.east);
		\draw [arrow] (out1) -- (stop);
	\end{tikzpicture}
	\captionof{figure}{Multiples of 3 or 5}
\end{center}

\section{Codes}
\begin{cpp}
	long Solution::sum_of_multiples(int cap)
	{
			long sum{};

			for(int i{1}; i < cap; ++i)
			{
					if(i % 3 == 0 || i % 5 == 0) sum += i;
				}
			return sum;
		}
\end{cpp}

\section{title}
\subsection{Problem Description}
\begin{tcolorbox}

\end{tcolorbox}
		

\section{Poker hands}
Problem 54

In the card game poker, a hand consists of five cards and are ranked, from lowest to highest, in the following way:

\rowcolors{0}{green!5}{white}
\noindent
\begin{tablebox}{\textsc{Poker Hands}}
	{
		\color{black!70} \sffamily
		\begin{tabularx}{\textwidth}{rX}
			High Card       & Highest value card                            \\
			One Pair        & Two cards of the same value                   \\
			Two Pairs       & Two different pairs                           \\
			Three of a Kind & Three cards of the same value                 \\
			Straight        & All cards are consecutive values              \\
			Flush           & All cards of the same suit                    \\
			Full House      & Three of a kind and a pair                    \\
			Four of a Kind  & Four cards of the same value                  \\
			Straight Flush  & All cards are consecutive values of same suit \\
			Royal Flush     & Ten, Jack, Queen, King, Ace, in same suit     \\
		\end{tabularx}
	}
\end{tablebox}

The cards are valued in the order:
2, 3, 4, 5, 6, 7, 8, 9, 10, Jack, Queen, King, Ace.

If two players have the same ranked hands then the rank made up of the highest value wins; for example, a pair of eights
beats a pair of fives (see example 1 below). But if two ranks tie, for example, both players have a pair of queens, then
highest cards in each hand are compared (see example 4 below); if the highest cards tie then the next highest cards are
compared, and so on.

Consider the following five hands dealt to two players:

\begin{tablebox}{\textsc{Examples}}
	\color{black!70} \sffamily
	\begin{tabularx}{\textwidth}{c X X c}
		{\textbf{Hand}} & {\textbf{Player 1}}                                             & {\textbf{Player 2}}                                              & {\textbf{Winner}} \\
		1               & 5H 5C 6S 7S KD \newline Pair of Fives                           & 2C 3S 8S 8D TD   \newline Pair of Eights                         & {Player 2}        \\
		2               & 5D 8C 9S JS AC\newline Highest card Ace                         & 2C 5C 7D 8S QH\newline Highest card Queen                        & {Player 1}        \\
		3               & 2D 9C AS AH AC\newline Three Aces                               & 3D 6D 7D TD QD\newline Flush with Diamonds                       & {Player 2}        \\
		4               & 4D 6S 9H QH QC\newline Pair of Queens\newline Highest card Nine & 3D 6D 7H QD QS\newline Pair of Queens\newline Highest card Seven & {Player 1}        \\
		5               & 2H 2D 4C 4D 4S\newline Full House With Three Fours              & 3C 3D 3S 9S 9D\newline Full House with Three Threes              & {Player 1}
	\end{tabularx}
\end{tablebox}
The file, poker.txt, contains one-thousand random hands dealt to two players. Each line of the file contains ten cards (separated by a single space): the first five are Player 1's cards and the last five are Player 2's cards. You can assume that all hands are valid (no invalid characters or repeated cards), each player's hand is in no specific order, and in each hand there is a clear winner.

How many hands does Player 1 win?

\rowcolors{0}{}{}

\section{Lychrel Numbers}

\subsection{Problem Description}
If we take \( 47 \), reverse and add,\( 47 + 74 = 121 \), which is palindromic.

Not all numbers produce palindromes so quickly. For example,
\begin{align*}
	349 + 943   & = 1292 \\
	1292 + 2921 & = 4213 \\
	4213 + 3124 & = 7337
\end{align*}

That is, \( 349 \) took three iterations to arrive at a palindrome.

Although no one has proved it yet, it is thought that some numbers, like \( 196 \)
, never produce a palindrome. A number that never forms a palindrome through the reverse and add process is called a
Lychrel number. Due to the theoretical nature of these numbers, and for the purpose of this problem, we shall assume
that a number is Lychrel until proven otherwise. In addition you are given that for every number below ten-thousand, it
will either (i) become a palindrome in less than fifty iterations, or, (ii) no one, with all the computing power that
exists, has managed so far to map it to a palindrome. In fact, \( 10677 \)
is the first number to be shown to require over fifty iterations before producing a palindrome:4668731596684224866951378664
(\( 53 \) iterations, \( 28 \) digits).

Surprisingly, there are palindromic numbers that are themselves Lychrel numbers; the first example is \( 4994 \).

How many Lychrel numbers are there below ten-thousand?

NOTE: Wording was modified slightly on 24 April 2007 to emphasise the theoretical nature of Lychrel numbers.

\subsection{Flow Chart}

\begin{tikzpicture}[node distance=2cm]
	\node [startstop] (start) {开始};
	\node [io, right of=start, xshift=4cm, text width=4cm] (io1) {输入一个考察的上限\\ 这里是10000};
	\node [operation, text width=2cm, below of=io1] (op1) {count = 0;\\ i = 10;};
	\node [decision, below of=op1, text width=2cm, yshift=-1cm] (dec1) {\( i < 上限 \)};
	\node [decision, below of=dec1, text width=2cm, yshift=-2cm] (dec2) {\( i \) is lychrel number?};
	\node [operation, below of=dec2, yshift=-1cm] (op2) {++count};
	\node [operation, right of=dec2, xshift= 4cm] (op3) {++i};
	\node [io, below of=op2] (io2) {输出count};
	\node [startstop, right of=io2, xshift=4cm] (stop) {结束};

	\draw[arrow] (start) -- (io1);
	\draw[arrow] (io1) -- (op1);
	\draw[arrow] (op1) -- (dec1);
	\draw[arrow] (dec1) -- node[right]{yes} (dec2);
	\draw[arrow] (dec1) -| node[right=.5cm, below=4cm]{no}([xshift=-2cm]io2.west) -- (io2);
	\draw[arrow] (dec2) -- node[right]{yes} (op2);
	\draw[arrow] (dec2) -- node[above]{no}(op3);
	\draw[arrow] (op3) |- (dec1);
	\draw[arrow] (io2) -- (stop);
\end{tikzpicture}

\chapter{Powerful Digit Sum}
\section{Description}
A googol ($10^{100}$) is a massive number: one followed by one-hundred zeros;
$100^{100}$is almost unimaginably large: one followed by two-hundred zeros. Despite their size, the sum of the digits in
each number is only .

Considering natural numbers of the form, $a^b$, where $a, b < 100$ what is the maximum digital sum?

\section{方根的渐进分数}
\subsection{问题描述}
\begin{tcolorbox}
可以证明,$\sqrt{2}$ 可以表示为一个无限的连分数:
\[
\sqrt{2} = 1 + \cfrac{1}{2 + \cfrac{1}{2 + \cfrac{1}{2 + \dots}}}
\]

通过展开前四次迭代,我们得到:
\[
1 + \cfrac{1}{2} = \cfrac{3}{2} = 1.5
\]
\[
1 + \cfrac{1}{2 + \cfrac{1}{2}} = \cfrac{7}{5} = 1.4
\]
\[
1 + \cfrac{1}{2 + \cfrac{1}{2 + \cfrac{1}{2}}} = \cfrac{17}{12} = 1.41666 \dots
\]
\[
1 + \cfrac{1}{2 + \cfrac{1}{2 + \cfrac{1}{2 + \cfrac{1}{2}}}} = \cfrac{41}{29} = 1.41379 \dots
\]

在第八次展开时,$\cfrac{1393}{985}$ 是第一个分子位数超过分母位数的例子。

问题:在前 1000 次展开中,有多少个分数的分子的位数多于分母的位数?
\end{tcolorbox}

\subsection{算法}


\subsection{答案}

\section{螺旋素数}\label{sec:problem58}
\subsection{问题描述}
\begin{tcolorbox}
	从数字 $1$ 开始,按照逆时针方向生成一个方形螺旋矩阵。例如,一个边长为 7 的方形螺旋矩阵如下:
	\[
		\begin{matrix}
			\red{37} & 36       & 35      & 34 & 33      & 32       & \red{31} \\
			38       & \red{17} & 16      & 15 & 14      & \red{13} & 30       \\
			39       & 18       & \red{5} & 4  & \red{3} & 12       & 29       \\
			40       & 19       & 6       & 1  & 2       & 11       & 28       \\
			41       & 20       & \red{7} & 8  & 9       & 10       & 27       \\
			42       & 21       & 22      & 23 & 24      & 25       & 26       \\
			\red{43} & 44       & 45      & 46 & 47      & 48       & 49       \\
		\end{matrix}
	\]

	其中,奇数平方数位于右下对角线上。在 7x7 的螺旋矩阵中,沿着对角线的 13 个数中有 8 个是质数,比例约为 $62\%$。

	问题:当螺旋矩阵的边长增加时,对角线上质数的比例首次下降到 $10\%$ 以下时,矩阵的边长是多少?
\end{tcolorbox}

\subsection{算法}
每增加一圈,加长加2,角点的步进等于当前边长-1。

这道题不能用埃氏筛法来解,效率比较低,因为随着边长的增加,角点数之间的间隔越来越大,判断角点数之间的数是效率的浪费。

\subsection{答案}
26241

\chapter{XOR Decryption}
\section{Description}
Problem 59
Each character on a computer is assigned a unique code and the preferred standard is ASCII (American Standard Code for
Information Interchange). For example, uppercase A = 65, asterisk (*) = 42, and lowercase k = 107.

A modern encryption method is to take a text file, convert the bytes to ASCII, then XOR each byte with a given value,
taken from a secret key. The advantage with the XOR function is that using the same encryption key on the cipher text,
restores the plain text; for example, 65 XOR 42 = 107, then 107 XOR 42 = 65.

For unbreakable encryption, the key is the same length as the plain text message, and the key is made up of random
bytes. The user would keep the encrypted message and the encryption key in different locations, and without both
"halves", it is impossible to decrypt the message.

Unfortunately, this method is impractical for most users, so the modified method is to use a password as a key. If the
password is shorter than the message, which is likely, the key is repeated cyclically throughout the message. The
balance for this method is using a sufficiently long password key for security, but short enough to be memorable.

Your task has been made easy, as the encryption key consists of three lower case characters. Using
\href{https://projecteuler.net/resources/documents/0059_cipher.txt}{0059\_cipher.txt}, a file containing the encrypted
ASCII codes, and the knowledge that the plain text must contain common English words, decrypt the message and find the
sum of the ASCII values in the original text.

\chapter{Prime Pair Sets}
\section{Description}
The primes 3, 7, 109, and 673, are quite remarkable. By taking any two primes and concatenating them in any order the
result will always be prime. For example, taking 7 and 109, both 7109 and 1097 are prime. The sum of these four primes,
792, represents the lowest sum for a set of four primes with this property.

Find the lowest sum for a set of five primes for which any two primes concatenate to produce another prime.

\section{循环多边形数}\label{sec:problem61}
\subsection{问题描述}
\begin{tcolorbox}
	三角形数、平方数、五边形数、六边形数、七边形数和八边形数都是多边形数,通过以下公式生成:

	\begin{itemize}
		\item 三角形数:$P_{3,n} = \dfrac{n(n+1)}{2}$ \quad $1, 3, 6, 10, 15, \dots$
		\item 平方数:$P_{4,n} = n^2$ \quad $1, 4, 9, 16, 25, \dots$
		\item 五边形数:$P_{5,n} = \dfrac{n(3n-1)}{2}$ \quad $1, 5, 12, 22, 35, \dots$
		\item 六边形数:$P_{6,n} = n(2n-1)$ \quad $1, 6, 15, 28, 45, \dots$
		\item 七边形数:$P_{7,n} = \dfrac{n(5n-3)}{2}$ \quad $1, 7, 18, 34, 55, \dots$
		\item 八边形数:$P_{8,n} = n(3n-2)$ \quad $1, 8, 21, 40, 65, \dots$
	\end{itemize}

	有序的三个四位数:$8128, 2882, 8281$,具有三个有趣的特性:

	\begin{enumerate}
		\item 这个集合是循环的,即每个数字的最后两位是下一个数字的前两位(包括最后一个数字与第一个数字)。
		\item 每种多边形类型:三角形($P_{3,127} = 8128$)、平方($P_{4,91} = 8281$)和五边形($P_{5,44} = 2882$),在集合中都有不同的数字。
		\item 这是唯一具有此属性的四位数集合。
	\end{enumerate}

	找出仅有的六个循环四位数的有序集合的和,其中每种多边形类型:三角形、平方、五边形、六边形、七边形和八边形,都由集合中的不同数字表示。

\end{tcolorbox}

\subsection{算法}
这道题可以用图的深度优先算法来解决。

\subsection{答案}
28684

\section{立方数排列}
\subsection{问题描述}
\begin{tcolorbox}
数字 \(41063625\) (\(345^3\)) 可以排列成两个其他的立方数:\(56623104\) (\(384^3\)) 和 \(66430125\) (\(405^3\))。实际上,\(41063625\) 是最小的立方数,它的数字排列可以产生正好三个也是立方数的排列。

找出最小的立方数,它的数字排列可以产生正好五个立方数的排列。
\end{tcolorbox}

\subsection{算法}


\subsection{答案}

\section{幂次数字计数}
\subsection{问题描述}
\begin{tcolorbox}
	5位数字的数 \( 16807 = 7^5 \) 也是一个五次方。同样地,9位数字的数 \( 134217728 = 8^9 \) 是一个九次方。

	有多少个 \( n \) 位正整数同时也是第 \( n \) 次方?
\end{tcolorbox}

\subsection{算法}
由于底数为10的任何次幂都会产生超过幂次本身位数的数字,因此,我们考虑的底数范围应限制在1到9之间。对于每一个特定的底数,我们从1次幂开始考察,随着幂次的增加,当幂次的值超过其位数时,就不可能再出现位数与幂次相等的情况。这意味着,对于每个底数,只有有限的幂次能够满足条件,即该数的位数与其幂次相等。

不过,这道题会用到大整数,直接计算会溢出。
\subsection{答案}
49

\section{奇数周期的平方根}
\subsection{问题描述}
\begin{tcolorbox}[breakable]
	所有平方根作为连分数表示时都是周期性的,可以写成以下形式:

	\[ \sqrt{N} = a_0 + \cfrac{1}{a_1 + \cfrac{1}{a_2 + \cfrac{1}{a_3 + \cdots}}} \]

	例如,让我们考虑 $\sqrt{23}$:

	\[ \sqrt{23} = 4 + \sqrt{23} - 4 = 4 + \cfrac{1}{\cfrac{1}{\sqrt{23} - 4}} = 4 + \cfrac{1}{1 + \cfrac{\sqrt{23} - 3}{7}} \]

	如果我们继续这个过程,我们会得到以下展开:

	\[ \sqrt{23} = 4 + \cfrac{1}{1 + \cfrac{1}{3 + \cfrac{1}{1 + \cfrac{1}{8 + \cdots}}}} \]

	这个过程可以总结如下:
	\begin{align*}
		a_0 & = 4, & \cfrac{1}{\sqrt{23} - 4}  = & \cfrac{\sqrt{23} + 4}{7}      = 1 + \cfrac{\sqrt{23} - 3}{7} \\
		a_1 & = 1, & \cfrac{7}{\sqrt{23} - 3}  = & \cfrac{7(\sqrt{23} + 3)}{14}  = 3 + \cfrac{\sqrt{23} - 3}{2} \\
		a_2 & = 3, & \cfrac{2}{\sqrt{23} - 3}  = & \cfrac{2(\sqrt{23} + 3)}{14}  = 1 + \cfrac{\sqrt{23} - 4}{7} \\
		a_3 & = 1, & \cfrac{7}{\sqrt{23} - 4}  = & \cfrac{7(\sqrt{23} + 4)}{7}   = 8 + \sqrt{23} - 4            \\
		a_4 & = 8, & \cfrac{1}{\sqrt{23} - 4}  = & \cfrac{\sqrt{23} + 4}{7} = 1 + \cfrac{\sqrt{23} - 3}{7}      \\
		a_5 & = 1, & \cfrac{7}{\sqrt{23} - 3}  = & \cfrac{7(\sqrt{23} + 3)}{14} = 3 + \cfrac{\sqrt{23} - 3}{2}  \\
		a_6 & = 3, & \cfrac{2}{\sqrt{23} - 3}  = & \cfrac{2(\sqrt{23} + 3)}{14} = 1 + \cfrac{\sqrt{23} - 4}{7}  \\
		a_7 & = 1, & \cfrac{7}{\sqrt{23} - 4}  = & \cfrac{7(\sqrt{23} + 4)}{7} = 8 + \sqrt{23} - 4              \\
	\end{align*}

	可以看出序列是重复的。为了简洁,我们使用记号 $\sqrt{23} = [4; (1, 3, 1, 8)]$,表示块 (1, 3, 1, 8) 无限重复。

	前十个连分数表示的(无理数)平方根是:
	\begin{align*}
		\sqrt{2}  & = [1; (2)], \quad \text{周期} = 1             \\
		\sqrt{3}  & = [1; (1, 2)], \quad \text{周期} = 2          \\
		\sqrt{5}  & = [2; (4)], \quad \text{周期} = 1             \\
		\sqrt{6}  & = [2; (2, 4)], \quad \text{周期} = 2          \\
		\sqrt{7}  & = [2; (1, 1, 1, 4)], \quad \text{周期} = 4    \\
		\sqrt{8}  & = [2; (1, 4)], \quad \text{周期} = 2          \\
		\sqrt{10} & = [3; (6)], \quad \text{周期} = 1             \\
		\sqrt{11} & = [3; (3, 6)], \quad \text{周期} = 2          \\
		\sqrt{12} & = [3; (2, 6)], \quad \text{周期} = 2          \\
		\sqrt{13} & = [3; (1, 1, 1, 1, 6)], \quad \text{周期} = 5 \\
	\end{align*}

	对于 $\sqrt{N} \leq 13$,正好有四个连分数表示具有奇数周期。

	对于 $\sqrt{N} \leq 10000$,有多少个连数表示具有奇数周期?

\end{tcolorbox}

\subsection{算法}
平方根(无理数)的连分数递推公式如下:
\begin{align*}
	\sqrt{d} & = \langle a_0, \overline{a_1, a_2, \cdots, a_i} \rangle \\
	a_0      & = \lfloor d \rfloor, c_0 = 0, q_0 = 1                   \\
	c_{i+1}  & = a_iq_i - c_i                                          \\
	q_{i+1}  & = \frac{d - c_i^2}{q_i}                                 \\
	a_i      & = \left\lfloor \frac{\sqrt{d} + c_i}{q_i} \right\rfloor
\end{align*}

\subsection{答案}
1322

\end{document}
