\section{扑克游戏}
\subsection{问题描述}
\begin{tcolorbox}[breakable]
	在扑克牌游戏扑克(Poker)中,一手牌由五张牌组成,并且按照以下方式从低到高排序:

	\begin{itemize}
		\item \textbf{高牌(High Card)}:最大的一张牌。
		\item \textbf{一对(One Pair)}:两张相同点数的牌。
		\item \textbf{两对(Two Pairs)}:两个不同点数的对子。
		\item \textbf{三条(Three of a Kind)}:三张相同点数的牌。
		\item \textbf{顺子(Straight)}:五张连续点数的牌。
		\item \textbf{同花(Flush)}:五张同花色的牌。
		\item \textbf{葫芦(Full House)}:三张相同点数的牌加一对。
		\item \textbf{四条(Four of a Kind)}:四张相同点数的牌。
		\item \textbf{同花顺(Straight Flush)}:五张同花色且连续点数的牌。
		\item \textbf{皇家同花顺(Royal Flush)}:10、J、Q、K、A同花色的牌。
	\end{itemize}

	牌的点数按照以下顺序从低到高排序:2, 3, 4, 5, 6, 7, 8, 9, 10, J, Q, K, A。

	如果两位玩家的手牌是同一种类型,则比较最大的一张牌。举例来说,成对的八(Pair of Eights)比成对的五(Pair of Fives)要大。如果两手牌中的最大牌相同,则比较次高的牌,依次类推。

	考虑以下两位玩家的五手牌对决:

	\[
		\begin{array}{|c|c|c|c|}
			\hline
			\text{手牌} & \text{玩家1}          & \text{玩家2}          & \text{胜者}   \\
			\hline
			1           & \texttt{5H 5C 6S 7S KD} & \texttt{2C 3S 8S 8D TD} & \text{玩家2 } \\
			2           & \texttt{5D 8C 9S JS AC} & \texttt{2C 5C 7D 8S QH} & \text{玩家1 } \\
			3           & \texttt{2D 9C AS AH AC} & \texttt{3D 6D 7D TD QD} & \text{玩家2 } \\
			4           & \texttt{4D 6S 9H QH QC} & \texttt{3D 6D 7H QD QS} & \text{玩家1 } \\
			5           & \texttt{2H 2D 4C 4D 4S} & \texttt{3C 3D 3S 9S 9D} & \text{玩家1 } \\
			\hline
		\end{array}
	\]

    文件 \href{https://projecteuler.net/resources/documents/0054_poker.txt}{poker.txt} 包含1000局随机生成的牌局,每局包含两位玩家的十张牌(以空格分隔),前五张是玩家1的手牌,后五张是玩家2的手牌。可以假设所有的手牌均合法(无无效字符或重复卡牌),每位玩家的手牌无顺序,并且在每局游戏中有明确的胜者。

	问题:玩家1赢得了多少局?

\end{tcolorbox}

\subsection{算法}
设计一个\texttt{PokerHand}类,比较大小用 \texttt{<=>}来判定大小:
\begin{enumerate}
    \item 如果手牌的种类不同,种类级别高的赢
    \item 如果手牌的种类相同,则依次比较
\end{enumerate}
要注意题目中不分花色的大小,另外皇家同花顺与同花顺没有必要作区分。

\subsection{答案}
