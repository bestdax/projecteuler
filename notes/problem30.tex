\section{数字的第五次幂}\label{sec:problem30}
\subsection{问题描述}
\begin{tcolorbox}

令人惊讶的是,只有三个数字可以写成它们各位数字的第四次幂之和:

\[
1634 = 1^4 + 6^4 + 3^4 + 4^4
\]

\[
8208 = 8^4 + 2^4 + 0^4 + 8^4
\]

\[
9474 = 9^4 + 4^4 + 7^4 + 4^4
\]

由于 $1 = 1^4$ 不是一个和,所以不包括在内。

这些数字的和是:

\[
1634 + 8208 + 9474 = 19316
\]

找出所有可以写成它们各位数字的第五次幂之和的数字的总和。
\end{tcolorbox}

\subsection{算法}
这道题实现起来并不难,主要是考虑判断的边界问题,这里的判断是到6位数为止,因为6位数最大为\num{999999},其数字的5次方之和为\num{354294},已经小于\num{999999}。而如果增加到7位数,最小的7位数是\num{1000000},而即使是\num{9999999}的数字5次方之和也只有\num{413343}。

另外一个可以优化的地方是可以将0--9所有数字的5次方计算出来,后面直接从数组中读取可以避免重复计算的问题。

\subsection{答案}
443839
