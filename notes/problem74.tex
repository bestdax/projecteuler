\section{数字阶乘链}
\subsection{问题描述}
\begin{tcolorbox}

	数字 $145$ 以其性质而闻名,即其数字的阶乘之和等于其本身:
	\[1! + 4! + 5! = 1 + 24 + 120 = 145.\]

	也许不太为人所知的是 $169$,它产生了最长的数字链,这些数字链最终会回到 $169$;结果表明只有三个这样的循环存在:
	\begin{align*}
		169 & \rightarrow 363601 \rightarrow 1454 \rightarrow 169 \\
		871 & \rightarrow 45361 \rightarrow 871                   \\
		872 & \rightarrow 45362 \rightarrow 872
	\end{align*}

	不难证明,每一个起始数字最终都会陷入循环。例如,

	\begin{align*}
		69  & \rightarrow 363600 \rightarrow 1454 \rightarrow 169 \rightarrow 363601 (\rightarrow 1454) \\
		78  & \rightarrow 45360 \rightarrow 871 \rightarrow 45361 (\rightarrow 871)                     \\
		540 & \rightarrow 145 (\rightarrow 145)
	\end{align*}

	以 $69$ 作为起始数字,产生了一个包含五个不重复项的链,但是以一百万以下起始数字的最长不重复链包含六十项。

	有多少个链,以一百万以下的起始数字,包含正好六十个不重复的项?
\end{tcolorbox}

\subsection{算法}
这道题解出答案并不难,关键是如何优化的问题。

用一个数组来存储当前数字可以形成的链的长度,这样在计算其他数字时可以查询数组,避免重复计算。

\subsection{答案}
402
